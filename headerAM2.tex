\newcommand{\mail}[3][blue]{\href{#2}{\color{#1}{#3}}}%

\renewenvironment{leftbar}[2][\hsize]
{
	\def\FrameCommand
	{
		{\color{#2}\vrule width 2pt}
		\hspace{0pt}
	}
	\MakeFramed{\hsize#1\advance\hsize-\width\FrameRestore}
}
{\endMakeFramed}

\newcommand\vertarrowbox[3][3ex]{%
	\begin{array}[t]{@{}c@{}} #2 \\
		\left\downarrow\vcenter{\hrule height #1}\right.\kern-\nulldelimiterspace\\
		\makebox[0pt]{\scriptsize#3}
	\end{array}%
}


\newenvironment{exbar}
{\begin{leftbar}{gray}}
{\end{leftbar}}

\newenvironment{dembar}
{\begin{leftbar}{black}}
	{\end{leftbar}}
	
\newenvironment{attbar}
{\begin{leftbar}{red}}
	{\end{leftbar}}


\everymath{\displaystyle}

\theoremstyle{definition}
\newtheorem{definition}{Definizione}[section]
\newtheorem{proposition}{Proposizione}[section]
\newtheorem{example}{Esempio}[section]

\theoremstyle{plain}
\newtheorem{theorem}{Teorema}[section]
\newtheorem{corollary}{Corollario}[theorem]
\newtheorem{lemma}[theorem]{Lemma}

\numberwithin{equation}{subsection}

\newcommand\distr
{\underset{\mbox{\tiny{triangolare}}}{\stackrel{\mbox{\tiny{disugualianza}}}{\leq}}}

\newcommand{\myarrow}[1][-45]{%
	\mathrel{%
		\text{$
			\begin{tikzpicture}[baseline = -0.5ex]
				\node[inner sep=0pt,outer sep=0pt,rotate = #1] (a) at (0,0)  {$\rightarrow{}$};
			\end{tikzpicture}
			$}%
	}%
}%

\makeatletter
\newcommand\mathcircled[1]{%
	\mathpalette\@mathcircled{#1}%
}
\newcommand\@mathcircled[2]{%
	\tikz[baseline=(math.base)] \node[draw,ellipse,inner sep=1pt] (math) {$\m@th#1#2$};%
}
\makeatother

