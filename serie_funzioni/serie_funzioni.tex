\section{Serie di funzioni}


\begin{definition}
	$(X,d)$ spazio metrico, $\{f_k\}_{k \in \mathbb{N}}, f_k : X \rightarrow \mathbb{R} $ (o $\mathbb{C}$) una successione di funzioni 
	
	$$\sum_{k=0}^{\infty} f_k, \qquad \lowercomment{\sum_{k=0}^{\infty}f_k(x)}
	{\text{è una serie numerica}} {\forall x \in X \text{ fissato}}$$. 
	
	Una serie di funzioni è una coppia 
	
	$$(\{f_k\}_{k \in \mathbb{N}},\{S_n\}_{n \in \mathbb{N}})$$ 
	
	dove $\{f_k\}_{k \in \mathbb{N}}, f_k: X \rightarrow \mathbb{R}$, è successione di funzioni e 
	
	$$\{S_n\}_{n \in \mathbb{N}}, \qquad S_n = \sum_{k=0}^{n}f_k \qquad {\color{blue} S_n(x) = \sum_{k=0}^{n} f_k(x) = f_0(x) +f_1(x) + \ldots + f_n(x)}$$
	
	è la successione delle ridotte. La serie è indicata con $\sum_{k=0}^{\infty}f_k$ o $\sum_{k=0}^{\infty}f_k (x)$.
\end{definition}


\begin{exbar}
	\begin{itemize}
		\item Serie esponenziale 
		\begin{gather*} 
			e^x= \sum_{k=0}^{\infty}\frac{x^k}{k!}, x \in \mathbb{R}
			\\
			X=\mathbb{R}, f_k (x)= \frac{ x^k}{k!}
		\end{gather*}
		
		\item Serie logaritmica 
		\begin{gather*} 
			\ln (1+x)= \sum_{k=0}^{\infty}(-1)^{k+1}\frac{x^k}{k}, x \in ]-1,1]=X
			\\
			f_k(x)=(-1)^{k+1}\frac{x^k}{k}
		\end{gather*}
	\end{itemize}
\end{exbar}


\begin{definition}
	Una serie di funzioni $\sum_{k=0}^{\infty} f_k$ si dice \textbf{puntualmente convergente} in $D \subseteq X$ se la sua successione delle ridotte converge puntualmente. In tal caso il limite puntuale si dice (funzione) somma della serie ed è indicato con $\sum_{k=0}^{\infty}f_k(x), x \in D$.
		
	Una serie di funzioni $\sum_{k=0}^{\infty}f_k$ si dice \textbf{uniformemente convergente} in $D \subseteq X$ se la sua successione delle ridotte converge uniformemente in $D$.
		
	\color{blue}{$\sum_{k=0}^{\infty}f_k$ \textbf{converge puntualmente} in $D$ se le serie numeriche $\sum_{k=0}^{\infty}f_k(x)$ convergono $\forall x \in D$.
	
	$\sum_{k=0}^{\infty}f_k$ \textbf{converge uniformemente} in $D$, 
	
	$$\sup_{x \in D}\bigg| \lowercomment{\sum_{k=0}^{\infty}f_k(x)}{S_n(x)}{} - \lowercomment{f(x)}{\text{somma delle}}{\text{serie}}|\xrightarrow{n \rightarrow+\infty}0$$}
\end{definition}


\begin{theorem} \textbf{di passaggio al limite sotto il segno di integrale}
	
	Sia $\sum_{k=0}^{\infty}f_k$ una serie di funzioni $f_k: [a,b]\rightarrow \mathbb{R}$, Riemann integrabili in $[a,b]$, uniformemente convergente in $[a,b]$ ad una funzione $f$, somma della serie. Allora $f$ è Riemann integrabile in $[a,b]$ e 
	\begin{equation*}
		\lowercomment{\int_{a}^{b}f(x)dx} {= \sum_{k=0}^{\infty} f_k(x)} {} =\sum_{k=0}^{\infty} \int_{a}^{b}f_k(x)dx
	\end{equation*}
	

	\begin{align*}
		\color{blue} \int_{a}^{b} \underbrace{\lim_{n\rightarrow +\infty}S_n(x)}_{f(x)} dx=
		&\color{blue} \underbrace{\lim_{n\rightarrow +\infty}\int_{a}^{b}S_n (x)dx}_{=}
		\\
		&\color{teal}\lim_{n\rightarrow +\infty}\int_{a}^{b}\sum_{k=0}^{n}f_k(x)dx=
		\\
		&\color{teal} =\lim_{n\rightarrow+\infty} \sum_{k=0}^{n} \int_{a}^{b} f_k(x)dx=
		\\
		&\color{teal} =\sum_{k=0}^{\infty}\int_{a}^{b}f_k(x)dx
	\end{align*}	
\end{theorem}


\begin{theorem} \textbf{di passaggio al limite sotto segno di derivata}
	
	$I \subseteq \mathbb{R}$ intervallo. $\{f_k\}_{k \in \mathbb{N}}, f_k: I \rightarrow \mathbb{R}$, successione di funzioni derivabili. Se 
	\begin{enumerate}
		\item $\exists\,\, x_0\in I \,\, \big|$ $\sum_{k=0}^{\infty}f_k(x_0)$ converge;
		
		\item la serie di funzioni $\sum_{k=0}^{\infty}f_k'$ converge uniformemente in $I$;
	\end{enumerate}
	
	allora la serie di funzioni $\sum_{k=0}^{\infty}f_k$ converge uniformemente in $I$ ad una funzione somma $f$ derivabile e tale che $f'(x)= \sum_{k=0}^{\infty}f_k'(x)$, $x \in I$, cioè $f'$ è la somma della serie delle derivate.
	
	\color{blue}{
		\begin{enumerate}
			\item $\exists x_0 \in I \,\, \big|$ $\{S_n(x_0)\}_{n \in \mathbb{N}},$ $S_n= \sum_{k=0}^{\infty}f_k$, converge
			
			\item $\{S_n'\}_{n \in \mathbb{N}}$ converge uniformemente in $I$ ad una funzione $g$.
		\end{enumerate}
		
		Allora $\{S_n\}_{n \in \mathbb{N}}$ converge uniformemente in $I$ ad una funzione derivabile $f$ tale che 
		
		$$f'=g=\lim_{n\rightarrow +\infty} S_n' =\lim_{n\rightarrow +\infty} \sum_{k=0}^{n} f_k'=\sum_{k=0}^{\infty}f_k'$$.
	}
\end{theorem}


\begin{proposition}
	\label{pr: pag 260}
	Sia $\sum_{k=0}^{\infty} f_k$ serie uniformemente convergente in $D \subseteq X$, $(X,d)$ spazio metrico. Allora 
	\begin{equation*}
		\lim_{k\rightarrow+\infty}\sup_{x \in D} |f_k(x)|=0,
	\end{equation*}
	
	cioè $f_k \rightrightarrows 0$ in $D$.
	
	\textcolor{blue}{Se $f_k \nrightrightarrows 0$ in $D$, la serie $\sum_{k=0}^{\infty}f_k$ non può convergere uniformemente.}
\end{proposition}


\begin{exbar}
	Serie esponenziale, $e^x= \sum_{k=0}^{\infty} \uppercomment{\frac{x^k}{k!}}{}{f_k(x)},$ $x \in \mathbb{R}$ che converge puntualmente in $\mathbb{R}$. Converge uniformemente in $\mathbb{R}$?
	\begin{center} 
		$\sup_{x \in \mathbb{R}}|f_k(x)|=\sup_{x \in \mathbb{R}}\frac{|x|^k}{k!}=+\infty \,\,\forall\,\, k$
		
		$\rightarrow f_k \nrightrightarrows 0$ in $\mathbb{R}$ 
		
		$\Rightarrow$ la serie non converge uniformemente in $\mathbb{R}$.
	\end{center} 
\end{exbar}


\begin{dembar}
	\textbf{Dimostrazione} della \textbf{Proposizione \ref{pr: pag 260}}
	
	$f:D \rightarrow \mathbb{R}$ somma della serie. Se $\{S_n\}_{n \in \mathbb{N}}$ è la successione delle ridotte, si ha 
	\begin{equation*}
		\lim_{n \rightarrow + \infty} \sup_{x \in D} |S_n(x)-f(x)|=0
	\end{equation*}
	
	perché $S_n \rightrightarrows f(x)$ in $D$.
	
	Fissato $\epsilon >0 \,\, \exists\,\, N >0\,\, \big|$ 
	\begin{itemize} 
		\item $n > N$
		
		$$\sup_{x \in D} |S_n(x)-f(x)|< \epsilon$$
		
		$\Rightarrow$ Se $n > N,$ $|S_n(x)-f(x)|< \epsilon \,\, \forall \,\, x \in D$
		
		\item $n > N +1$
		
		$$|f_n(x)|=|\lowercomment{S_n(x)}{=}{\sum_{k=0}^{n} f_k(x)}-\lowercomment{S_{n-1}(x)}{=}{\sum_{k=0}^{n-1} f_k(x)}| \leq \underbrace{|S_n (x)-f(x)|}_{\color{blue} <\epsilon, \; n>N+1}+ \underbrace{|S_{n-1}(x)-f(x)|}_{\color{blue} <\epsilon \text{ perché } n-1>N} < 2 \epsilon \,\, \forall \,\, x \in D$$
		
		Se $n > N+1 \Rightarrow \sup_{x \in D}|f_n(x)| < 2 \epsilon \Rightarrow f_n \rightrightarrows 0$ in $D$. $\qquad\square$
	\end{itemize}
\end{dembar}


\begin{definition}
	Una serie di funzioni $\sum_{k=0}^{\infty}f_k$ si dice \textbf{totalmente convergente} in $D \subseteq X$ se converge la serie numerica 
	\begin{equation*}
		\sum_{k=0}^{\infty}\sup_{x\in D}|f_k(x)|
	\end{equation*}
\end{definition}


\textbf{Osservazione:}

Se $\exists\,\, \{a_k\}_{k \in \mathbb{N}}\subseteq \mathbb{R}$ tale che 
\begin{enumerate}
	\item $|f_k(x)|\leq a_k \,\, \forall \,\, x \in D$ e $\forall\,\, k \in \mathbb{N} $
	\item $\sum_{k=0}^{\infty}a_k$ converge 
\end{enumerate}

allora $\sum_{k=0}^{\infty}f_k$ converge totalmente in $D$.

Infatti, se $|f_k(x)|\leq a_k \,\,\forall \,\,x \in D$ $\Rightarrow \sup_{x \in D} |f_k (x)|\leq a_k$ $\forall\,\, k \in \mathbb{N} \Rightarrow$ la serie $\sum_{k=0}^{\infty} \sup_{x \in D}|f_k(x)|$ converge per il criterio del confronto.


\begin{theorem} \textbf{Criterio di Weierstrass}
	\label{th: pag 263}
	Sia $\sum_{k=0}^{\infty}f_k$ serie di funzioni totalmente convergente in $ D \subseteq X$, $ (X,d)$ spazio metrico. Allora $\sum_{k=0}^{\infty}f_k$ converge uniformemente in $D$.	
\end{theorem}


\begin{dembar}
	\textbf{Dimostrazione} del \textbf{Teorema \ref{th: pag 263}}
	
	$\sum_{k=0}^{\infty}\sup_{y \in D} |f_k(y)|$ converge.
	
	Siccome $|f_k(x)|\leq \sup_{y \in D}|f_k(y)|$ $\forall\,\, x \in D$ per il criterio del confronto $\sum_{k=0}^{\infty}|f_k(x)|$ converge $\forall \,\, x \in D$ e quindi, per il criterio di convergenza assoluta, converge $\forall\,\, x \in D$ la serie $\sum_{k=0}^{\infty} f_k(x)$.
	
	Sia $f(x)=\sum_{k=0}^{\infty}f_k(x)$ la sua somma, $x \in D$.
	
	Stimiamo $|S_n(x)-f(x)|$ dove $S_n(x)=\sum_{k=0}^{n}f_k(x)$.
	
	$\sum_{k=n+1}^{\infty}f_k(x)=f(x)-S_n(x)$ è ben definita $\forall \,\, n$, perché è il resto di una serie convergente
	
	\color{blue}{($\exists$ finito $\lim_{p \rightarrow+\infty}\sum_{k=n+1}^{p}f_k(x)$)}.
	
	\begin{align*} 
		|S_n(x)-f(x)|=|\sum_{k=n+1}^{\infty}f_k(x)|\leq \sum_{k=n+1}^{\infty}|f_k(x)| \leq
		\\
		\leq \lowercomment{\sum_{k=n+1}^{\infty} \sup_{y \in D}|f_k(y)|} {\color{blue} 0 \; n \to + \infty} {\color{red} \text{non dipende da } x\in D}
	\end{align*}
	
	perché è il resto di una serie convergente 
	
	\begin{center}
	$\Rightarrow \underbrace{\sup_{x \in D}|S_n(x)-f(x)|}_{\color{red}0 \; n \to + \infty} \leq \lowercomment{\sum_{k=n+1}^{\infty}\sup_{y \in D} |f_k(y)|}{0 \; n \to + \infty}{}$
	
	$\Rightarrow S_n \rightrightarrows f$ in $D$. $\qquad\square$
	\end{center}
\end{dembar}


\begin{exbar}
\begin{example}
	La convergenza totale non è equivalente a quella uniforme, cioè una serie può convergere uniformemente senza convergere totalmente.
	
	Consideriamo 
	\begin{equation*}
		\sum_{k=1}^{\infty}(-1)^k \frac{|x|^k}{k},\,\, x \in [-1,1]    
	\end{equation*}
	
	la cui somma è $-\ln (1+|x|)$. \'E una serie di Leibniz. Detta $S_n(x)$ la sua ridotta $n$-esima
	\begin{center} 
		$|S_n(x)-(-\ln (1+|x|))|\leq \frac{|x|^{n+1}}{n+1}\leq \frac{1}{n+1}\,\, \forall x \in [-1,1]$
		
		$\sup_{x \in [-1,1]}|S_n(x)-(-\ln (1+|x|))| \leq \frac{1}{n+1}\xrightarrow{n \rightarrow +\infty} 0$
		
		$\Rightarrow$ la serie converge uniformemente in $[-1,1]$
	\end{center} 
	
	Studiamo la convergenza totale. 
	
	$$\sum_{k=1}^{\infty}\sup_{x \in [-1,1]}|(-1)^k\frac{|x|^k}{k}|= \sum_{k=1}^{\infty} \frac{1}{k}=+\infty$$
	
	cioè la serie di funzioni non converge totalmente in $[-1,1]$. 
\end{example}
\end{exbar}


\begin{exbar}
\begin{example}
	Studiamo la convergenza puntuale, uniforme e totale di 
	\begin{equation*}
		\sum_{k=0}^{\infty}z^k,\,\, z\in \mathbb{C}.    
	\end{equation*}
	
	E' una serie geometrica di ragione $z \Rightarrow$ converge $\Leftrightarrow |z|<1$ e in tal caso la sua funzione somma è $f(z)=\frac{1}{1-z}$.
	
	Insieme di convergenza puntuale è 
	
	$$B_1(0)=\{z \in \mathbb{C}\,\,|\,\, |z|< 1\}$$
	
	La ridotta $n$-esima è 
	
	$$S_n(z)=\frac{1-z^{n+1}}{1-z}$$
	
	$$|S_n(z)-f(z)|=\left|\frac{1-z^{n+1}}{1-z}-\frac{1}{1-z} \right|=\frac{|z|^{n+1}}{|1-z|}$$
	
	$$\sup_{z \in B_1(0)}|S_n(z)-f(z)|=+\infty \,\,\, \forall n$$ 
	
	$\Rightarrow$ la serie di funzioni non converge uniformemente in $B_1(0)$.
	
	Studiamo la convergenza uniforme in $B_\delta(0)=\{z \in \mathbb{C}\,\, \big|$ $|z|< \delta\}$, con $0 < \delta< 1$ fissato.
	
	\begin{gather*} 
		|f_k(z)|=|z|^k< \delta^k \,\, \forall z \in B_\delta(0)
		\\
		\sup_{z \in B_\delta(0)}|f_k(z)|\leq \uppercomment{\delta^k}{\color{red}\text{termine generale di una}}{\color{red}\text{serie geometrica convergente}} 
		\\
		\Rightarrow \sum_{k=0}^{\infty}\sup_{z\in B_\delta(0)}|f_k(z)| \text{ converge}
	\end{gather*}
	
	$\Rightarrow$ la serie $\sum_{k=0}^{\infty} z^k$ converge totalmente, e quindi uniformemente, in $B_\delta(0)$, per ogni $0 < \delta < 1$ fissato.
\end{example}
\end{exbar}


\begin{exbar}
\begin{example}
	Consideriamo la serie esponenziale
	\begin{equation*}
		e^z= \sum_{k=0}^{\infty} \frac{z^k}{k!}, \,\,z \in \mathbb{C}.    
	\end{equation*}
	
	Questa serie non converge uniformemente in $\mathbb{C}$.
	
	$$\sup_{z\in\mathbb{C}} \left|\frac{z^k}{k!}\right|=\sup_{z\in\mathbb{C}}\frac{|z|^k}{k!}=+\infty\,\, \forall k \geq 1$$
	
	$\Rightarrow$ detta $f_k(z)=\frac{z^k}{k!}$, si ha $f_k \nrightrightarrows 0$ in $\mathbb{C}$.
	
	Sia $M >0 $ e consideriamo 
	
	$$B_M(0)=\{z \in \mathbb{C}\,\, | \,\, |z|< M\}$$
	
	$$\sup_{z \in B_M(0)}|f_k(z)|=\sup_{z \in M}\frac{|z|^k}{k!}=\frac{M^k}{k!}$$ 
	
	e $\sum_{k=0}^{\infty} \frac{M^k}{k!}=e^M$, cioè $\sum_{k=0}^{\infty}\frac{M^k}{k!}$ converge
	
	$\Rightarrow \sum_{k=0}^{\infty}\sup_{z \in B_M(0)}|f_k(z)|$ converge $\Rightarrow \sum_{k=0}^{\infty}\frac{z^k}{k!}$ converge totalmente, e quindi uniformemente, in $B_M(0) \,\, \forall M >0$ fissato.
	
	\color{blue}{Detta $S_n(x)=\sum_{k=0}^{n}\frac{x^k}{k!},\,\, x\in \mathbb{R}$, si può stimare $\sup_{x \in \mathbb{R}}|S_n(x)-e^x|$, con $e^x=\sum_{k=0}^{\infty}\frac{x^k}{k!}$.
		
	\begin{center} 
		$\sup_{ x \in \mathbb{R}}| \underbrace{\sum_{k=0}^{n} \frac{x^k}{k!}}_{\genfrac{}{}{0pt}{}{\color{teal}\text{polinomio di}}{\color{teal}\text{grado } n}}
		-e^x|=+\infty$ 
		
		perché $\lim_{x \rightarrow +\infty}|\sum_{k=0}^{n}\frac{x^k}{k!}-e^x|=+\infty$
	\end{center}
	}
\end{example}
\end{exbar}


\begin{exbar}
\begin{example}
	Esprimere $\int_{0}^{1}e^{-x^2}dx$ come somma di una serie e darne una stima del valore con $8$ cifre decimali esatte.
	\begin{gather*}
		e^{-x^2}=\sum_{k=0}^{\infty}\frac{(-x^2)^k}{k!}=\sum_{k=0}^{\infty}(-1)^k \frac{x^{2k}}{k!}
		\\
		\sum_{k=0}^{\infty}\sup_{x \in [0,1]}|(-1)^k \frac{x^{2k}}{k!}|=\sum_{k=0}^{\infty} \frac{1}{k!},
	\end{gather*}
		 
	che è serie convergente
	
	$\Rightarrow \sum_{k=0}^{\infty}(-1)^k \frac{x^{2k}}{k!}$ converge totalmente, e quindi uniformemente in $[0,1]$.
	
	Allora
	
	$$\int_{0}^{1}e^{-x^2}dx=\int_{0}^{1}\sum_{k=0}^{\infty}(-1)^k\frac{x^{2k}}{k!}dx=\sum_{k=0}^{\infty} (-1)^k \int_{0}^{1}\frac{x^{2k}}{k!}dx=\sum_{k=0}^{\infty}(-1)^k \frac{1}{(2k+1)k!}$$
	
	$$S_n=\sum_{k=0}^{n}(-1)^k \frac{1}{(2k+1)k!}$$
	
	$$\left|S_n-\int_{0}^{1}e^{-x^2}dx \right| \leq \frac{1}{(2n+3)(n+1)!}$$ 
	
	scelgo $n$ in modo che $\frac{1}{(2m+3)(n+1)!}<10^{-8}$
	
	Preso $n=10$, si ha che
	
	$$\sum_{k=0}^{10}(-1)^k\frac{1}{(2k+1)k!}=1-\frac{1}{3}+\frac{1}{10}+...+\frac{1}{21\cdot10!}$$ 
	
	è una stima di $\int_{0}^{1}e^{-x^2}dx$ con $8$ cifre decimali esatte.
\end{example}
\end{exbar}


\begin{exbar}
\begin{example}
	Sia data la serie di funzioni
	\begin{equation*}
		\sum_{k=1}^{\infty}\frac{\sin(kx)}{k^x},\,\,\, x \geq 0
	\end{equation*}
	
	\begin{enumerate}
		\item Trovare l'insieme di convergenza puntuale
		\item Dimostrare che la serie converge totalmente in tutti gli intervali del tipo $[a,+\infty[$ con $a > 1$, ma non in $]1,+\infty[$
		\item Dimostrare che la funzione somma è derivabile in $]2,+\infty[$
	\end{enumerate}
	
	\begin{center} $\sim \circ \sim $ \end{center}
	 
	\begin{equation*}
		f_k(x)=\frac{\sin(kx)}{k^x},\,\,\, k \geq 1, \,\, x \geq 0
	\end{equation*}
	
	%====================================================================================
	
	\begin{enumerate}
		\item $f_k(0)=0\,\, \forall\,\, k \geq 1 $ e quindi $\sum_{k=1}^{\infty} f_k(0)=0$, $x >0$, $\sum_{k=1}^{\infty} \frac{\sin(kx)}{k^x}$.
		
		Osserviamo che $\{\frac{1}{k^x}\}_{k \geq 1}$ è monotona decrescente e infinitesima.
		
		Prendiamo la successione delle ridotte di $\sum_{k=1}^{\infty}\sin(kx)$
		\begin{gather*} 
			S_n(x)=\sum_{k=1}^{n}\sin(kx)= \lowercomment{\Im \left(\sum_{k=1}^{n}e^{ikx}\right) } {\text{ridotta n-esima di una serie}} {\text{geometrica di ragione } e^{ix}} = \Im \left( \frac{1-e^{e^{i(k+1)x}}}{1-e^{ix}}-1\right)
			\\
			\text{ con } x \neq 2j\pi,\,\,\, j \geq 1, \,\,\, , j \in \mathbb{Z}
			\\
			|S_n(x)|\leq |\frac{1-e^{i(n+1)x}}{1-e^{ix}}-1|\leq \frac{2}{|1-e^{ix}|}+1, \qquad x \neq 2j\pi, \qquad j \geq 1
		\end{gather*}
		
		$\{S_n(x)\}_{n \geq 1}$ è limitata $\forall\,\, x >0,\,\, x \neq 2j\pi \Rightarrow \sum_{k=1}^{\infty}\frac{\sin(kx)}{k^x}$ converge puntualmente per $x \geq 0, x \neq 2j\pi$, $j \geq 1$ per il criterio di Dirichlet.
		
		Se $x = 2j \pi$ allora $f_k(x)=f_k(2j\pi)=0 = \frac{\sin(2jk\pi)}{k^{2j\pi}} \,\, \forall k$ $\Rightarrow \sum_{k=1}^{\infty} f_k(2j\pi)=0$ $\Rightarrow$ la serie converge puntualmente $\forall x \geq 0$.
		
		\item $a > 1$, $x \in [a,+\infty[$, $x\geq a $.
		
		Dobbiamo studiare la convergenza della serie numerica 
		
		$$\sum_{k=1}^{\infty}\sup_{x \geq a} \left| \frac{\sin(kx)}{k^x} \right|$$
		
		se $x \geq a$ si ha $\left| \frac{\sin(kx)}{k^x} \right| \leq \frac{1}{k^x}\leq \frac{1}{k^a}$ e $\sum_{k=1}^{\infty}\frac{1}{k^a}$ converge perché $a > 1$
		
		$$\Rightarrow \sup_{x \geq a} \left|\frac{\sin(kx)}{k^x} \right| \leq \frac{1}{k^a}$$ 
		
		e quindi la serie converge totalmente, e quindi uniformemente in $[a,+\infty[$.
		
		Dimostriamo che non converge totalmente in $]1,+\infty[$, cioè che la serie numerica 
		
		\begin{center} 
			$\sum_{k=1}^{\infty}\sup_{x >1}|\frac{\sin(kx)}{k^x}|$ diverge
			
			$\sup_{x >1}|\frac{\sin(kx)}{k^x}|\geq \lim_{x \rightarrow 1^+}|\frac{\sin(kx)}{k^x}|=\frac{|\sin (k)|}{k}$
		\end{center}
		
		Se dimostro che $\sum_{k=1}^{\infty}\frac{|\sin(k)|}{k}$ diverge, posso concludere per il criterio del confronto.
		
		$$\frac{\sin(k+1)}{k+1}=\frac{\sin (k)}{k+1}\cos (1) +\frac{\cos(k)}{k+1}\sin(1)$$
		
		$$\frac{\cos(k)}{k+1}=\frac{1}{\sin(1)}\left[\frac{\sin(k+1)}{k+1} -\frac{\sin(k)}{k+1} \cos(1) \right]$$
		
		$$|\frac{\sin(x)}{k+1}| \sim \frac{|\sin(k)|}{k}$$
		
		Se $\sum_{k=1}^{\infty} \frac{|\sin(k)|}{k}$ converge,  allora converge $\sum_{k=1}^{\infty} \frac{|\sin(k)|}{k+1}$ e, banalmente, anche \\%riq mal
		$\sum_{k=1}^{\infty} \left|\frac{\sin(k+1)}{k+1} \right|$
		
		$\Rightarrow \sum_{k=1}^{\infty} \frac{|\cos(k)|}{k+1}$ converge, perché $\sum_{k=1}^{\infty} \frac{\cos(k)}{k+1}$ è somma di serie assolutamente convergenti.
		
		Ma $\frac{|\cos(k)|}{k+1} \sim \frac{|\cos(k)|}{k}$ $\Rightarrow \sum_{k=1}^{\infty}\frac{|\cos(k)|}{k}$ converge, cioè, se $\sum_{k=1}^{\infty} \frac{\sin{k}}{k}$ converge assolutamente, anche $\sum_{k=1}^{\infty} \frac{\cos(k)}{k}$ converge assolutamente, e questo è assurdo perché \\%riq mal
		$\frac{e^{ik}}{k}=\frac{\cos(k)}{k} +i \frac{\sin(k)}{k}$ e quindi $\sum_{k=1}^{\infty}\frac{e^{ik}}{k}$ convergerebbe assolutamente. 
		
		Ma $\sum_{k=1}^{\infty} \left| \frac{e^{ik}}{k} \right|=\sum_{k=1}^{\infty}\frac{1}{k}$ diverge.
		
		\item $f(x)=\sum_{k=1}^{\infty} \uppercomment{\frac{\sin(kx)}{k^x}}{}{f_k(x)}, \,\,\, x \geq 0$.
		
		Devo studiare la derivabilità di $f$ in $]2,+\infty[$. Se la serie derivata 
		
		$$f(x)=\sum_{k=1}^{\infty} f_k'(x)$$
		
		converge uniformemente in $]2,+\infty[$, il teorema di derivabilità per serie permette di concludere che $f$ è derivabile.
		
		\begin{center} 
			$ f_k'(x)=\frac{k \cos(kx)-(\ln k) \sin(kx)}{k^x}$
		
			$|f_k'(x)|\leq \frac{k+\ln k}{k^x} \lowercomment{\leq}{x>2}{} \frac{k+\ln k}{k^2} \sim \frac{1}{k}$ che però diverge.
		\end{center}
		
		Consideriamo dunque un intervallo del tipo $[b,+\infty[$ con $b >2$
		
		$$|f_k'(x)|\leq \frac{k+\ln k}{k^x} \lowercomment{\leq}{x\geq b}{} \frac{k+\ln k}{k^b}$$, 
		
		$$\sup_{ x \geq b}|f_k'(x)|\leq \frac{k +\ln k}{k^b}\sim \frac{1}{k^{b-1}},$$
		
		che è il termine generale di una serie convergente perché $b-1  > 1$ $\Rightarrow$ la serie $\sum_{k=1}^{\infty}f_k'(x)$ converge totalmente, e quindi uniformemente, in $[b,+\infty[$ $\Rightarrow f$ è derivabile in $[b,+\infty[ \forall b > 2$.
		
		Fissato $x_0 > 2$, posso concludere che $f$ è derivabile in $x_0$.
		
		\color{blue}{Quindi $f$ è derivabile in $]2,+\infty[$}?\\
		Certo, perché fissato $2 < b < x_0$, $f$ è derivabile in $[b,+\infty[$ e quindi anche in $x_0$.
	\end{enumerate}
\end{example}
\end{exbar}


\begin{exbar}
\begin{example}
	Studiamo la convergenza puntuale, uniforme e totale della serie di funzioni 
	\begin{equation*}
		\sum_{k=2}^{\infty}\big[ \underbrace{\frac{k^y+1}{k^{x+2}(\ln k)^2}}_{\color{blue}>0} + \underbrace{\left( \frac{x}{y} \right)^k}_{\color{blue}>0} \bigg], \,\,\, x,y >0
	\end{equation*} 
	
	La serie è somma di due serie a termini positivi e quindi converge puntualmente $\Leftrightarrow$ le due serie 
	
	$$\sum_{k=2}^{\infty} \frac{k^y+1}{k^{x+2}(\ln k)^2} \text{ e } \sum_{k=2}^{\infty}  \left( \frac{x}{y} \right)^k$$
	
	convergono puntualmente.
		
	Studiamo $\sum_{k=2}^{\infty} \frac{k^y+1}{k^{x+2}(\ln k)^2}$, $x,y >0$
	
	$$\frac{k^y+2}{k^{x+2}(\ln k)^2} \sim \frac{k^y}{k^{x+2}(\ln k)^2}=\frac{1}{k^{x-y+2}(\ln k)^2}$$
	
	che è il termine generale di una serie convergente $\Leftrightarrow x-y+2 \geq 1 \Leftrightarrow x-y \geq -1$
	
	Studiamo $\sum_{k=2}^{\infty} \left(\frac{x}{y}\right)^k$, $x,y>0$, che è una serie geometrica di ragione $\frac{x}{y}$, che converge $\Leftrightarrow |\frac{x}{y}|< 1 \Leftrightarrow x <y$.
	
	La serie di partenza converge puntualmente $\Leftrightarrow 0<x<y\leq x+1$

	\segnaposto

	Studiamo la convergenza uniforme nell'insieme di convergenza puntuale. Osserviamo che 
	
	$$\sup_{0<x< y\leq x+1}|\frac{k^y+1}{k^{x+2}(\ln k)^2}+\left( \frac{x}{y}\right)^k|\geq  \sup_{0<x< y\leq x+1} \left( \frac{x}{y}\right)^k=1\,\, \forall k$$ 
	
	perché il rapporto $\frac{x}{y}$ può essere preso vicino a $1$ quanto vogliamo.
	
	In particolare, 
	$$\sup_{0<x< y\leq x+1}|\frac{k^y+1}{k^{x+2}(\ln k)^2}+\left( \frac{x}{y}\right)^k|\nrightarrow 0 \text{ per } k \rightarrow +\infty$$
	
	e quindi la serie non converge uniformemente nell'insieme di convergenza puntuale.
	
	Fissiamo $0 < a  < 1$  e studiamo la convergenza totale nell'insieme 
	
	$$\underbrace{0 <x \leq a y \leq a(x+1)}_{\color{blue} \left| \frac{x}{y} \right| \leq a < 1}$$
	
	\segnaposto

	Dobbiamo stimare 
	
	$$\sup_{0 < \leq ay \leq a(y+1)} \left| \frac{k^y+1}{k^{x+2}(\ln k)^2} +\left(\frac{x}{y}\right)^k \right|$$
	
	\begin{align*} 
		\left| \frac{k^y+1}{k^{x+2}(\ln k)^2} +\left(\frac{x}{y}\right)^k \right|
		&= \frac{k^y+1}{k^{x+2}(\ln k)^2} +\left(\frac{x}{y}\right)^k \lowercomment{\leq}{y\leq x+1}{0<\frac{x}{y}\leq a}  \frac{k^{x+1}+\uppercomment{1}{}{\leq k^{x+1}}}{k^{x+2}(\ln k)^2}+a^k \leq
		\\
		&\leq 2\frac{k^{x+1}}{k^{x+2}(\ln k)^2}+a^k =\frac{2}{k(\ln k)^2}+a^k, 
		\\
		\forall \,\, x,y >0, & \qquad 0 < x \leq ay\leq a(x+1)
	\end{align*}

	La serie $\sum_{k=2}^{\infty}\left[\frac{2}{k(\ln k)^2}+a^k\right]$ converge perché entrambe le serie 
	
	$$\sum_{k=1}^{\infty}\frac{2}{k(\ln k)^2} \text{ e } \sum_{k=2}^{\infty}a^k$$
	
	convergono $\Rightarrow$ la serie di partenza converge totalmente, e quindi uniformemente negli insiemi del tipo 
	
	$$\{(a,y)\in \mathbb{R}^2|0<x\leq ay, y \leq x+1\}$$
	
	con $0 < a<1$ fissato.
\end{example}
\end{exbar}


\subsection{Serie di potenze}

\begin{definition}
	Una serie di potenze centrata nel punto $z_0 \in \mathbb{C}$ è una serie di funzioni del tipo 
	\begin{equation*}
		\sum_{k=0}^{\infty} a_k(z-z_0)^k,
	\end{equation*}
	
	dove $\{a_k\}_{k\in\mathbb{N}}$ è una successione di numeri complessi.
\end{definition}


\begin{exbar}
	Serie esponenziale
	\begin{equation*}
		e^z=\sum_{k=0}^{\infty}\frac{z^k}{k!},\,\,\,\, z_0=0,\,\,\, a_k=\frac{1}{k!}
	\end{equation*}
\end{exbar}


\begin{attbar}
	\textbf{Osservazione:}
	
	Non è restrittivo assumere $z_0=0$. Infatti, posto $w=z-z_0$, la serie 
	$$\sum_{k=0}^{\infty} a_k(z-z_0)^k$$ 
	
	si riscrive come 
	
	$$\sum_{k=0}^{\infty} a_k w^k,$$
	
	cioè come una serie di potenze di centro $w_0=0$.
\end{attbar}


\begin{theorem} \textbf{Criterio di Cauchy-Hadamard}
	\label{th: pag 285}
	Sia data la serie di potenze
	\begin{equation*}
		\sum_{k=0}^{\infty} a_kz^k,\,\,\, z \in \mathbb{C}
	\end{equation*}
	
	e sia $R\in [0,+\infty]$ la quantità definita da 
	\begin{equation*}
		\frac{1}{R}=\limsup_{k \rightarrow+\infty} \sqrt[k]{|a_k|},
	\end{equation*}
	
	dove, se
	\begin{equation*}
		\limsup_{k \rightarrow+\infty} \sqrt[k]{|a_k|} =0 \Rightarrow R=+\infty
	\end{equation*}
	
	e se
	\begin{equation*}
		\limsup_{k \rightarrow+\infty} \sqrt[k]{|a_k|} =+\infty \Rightarrow R=0.
	\end{equation*}

	Allora
	\begin{enumerate}
		\item la serie converge assolutamente in $B_k(0)=\{z\in\mathbb{C} \big|$ $|z|<R\}$, detto disco o cerchio di convergenza;
		
		\item la serie converge totalmente, e quindi uniformemente, in $B_\delta (0)=\{z \in \mathbb{C} \big|$ $|z|< \delta\}$ per ogni $0< \delta < R$;
		
		\item la serie non converge per $|z|>R$.
	\end{enumerate}
	
	$R$ è detto raggio di convergenza della serie.
\end{theorem}


\begin{attbar}
	\textbf{Osservazione:}
	
	Il teorema non fornisce informazioni sulla convergenza della serie per $|z|=R$.
\end{attbar}


\begin{definition}
	Data una successione $\{\alpha_k\}_{k\in\mathbb{N}}\subseteq \mathbb{R}$, 
	\begin{equation*}
		\limsup_{k\rightarrow+\infty} \alpha_k=\inf_{k\in\mathbb{N}}\left(\sup_{\ell>k} \alpha_\ell\right)=\lim_{k \rightarrow+\infty}\left(\sup_{\ell>k}\alpha_\ell\right)
	\end{equation*}
	
	ed esiste sempre.
	
	\color{red}{Se $\exists \lim_{k\rightarrow+\infty}\alpha_k$, allora $\limsup_{k\rightarrow+\infty}\alpha_k=\lim_{k\rightarrow+\infty}\alpha_k$.
		
	Nel criterio di Hadamard, se $\exists \lim_{k\rightarrow+\infty}\sqrt[k]{|\alpha_k|}$, allora $\frac{1}{R}=\lim_{k\rightarrow+\infty}\sqrt[k]{|a_k|}$}
\end{definition}


\textbf{Osservazione:}

Si ricorda che, se $\lim_{k\rightarrow+\infty}|\frac{a_k+1}{a_k}|=\ell$, allora $\lim_{k\rightarrow+\infty}\sqrt[k]{|a_k|}=\ell$.


\begin{dembar}
	\textbf{Dimostrazione} del \textbf{Teorema \ref{th: pag 285}}
	
	Lo dimostriamo assumendo che esista $\lim_{k\rightarrow+\infty}\sqrt[k]{|a_k|}=\frac{1}{R}$.
	\begin{enumerate}
		\item Dobbiamo dimostrare che la serie $\sum_{k=0}^{\infty}|a_kz^k|=\sum_{k=0}^{\infty}|a_k||z|^k$ converge se $|z|<R$.
		
		Applichiamo il criterio della radice e calcoliamo 
		
		$$\lim_{k \rightarrow+\infty}\sqrt[k]{|a_k||z|^k}=\frac{|z|}{R}$$
		
		Se $\frac{|z|}{R}<1$, la serie converge.
		
		\item Dobbiamo dimostrare che, fissato $0<\delta<k$, la serie 
		
		$$\sum_{k=0}^{\infty}\sup_{|z|<\delta} |a_kz^k|$$ 
		
		converge.
		
		Se $|z|<\delta$ $\Rightarrow |a_kz^k|=|a_k||z|^k<|a_k|\delta^k$ e la serie $\sum_{k=0}^{\infty}|a_k|\delta^k$ converge perché\\%riq mal
		$\lim_{k \rightarrow+\infty} \sqrt[k]{|a_k|\delta^k}=\frac{\delta}{R}<1$. 
		
		Allora $\sup_{|z|<\delta}|a_kz^k|<|a_k|\delta^k\,\, \forall k$ e quindi deduco la convergenza totale della serie in $B_{\delta}(0)$.
		
		\item Se $\frac{|z|}{R}>1$, il termine generale della serie non è infinitesimo, e quindi la serie non converge.
	\end{enumerate}
\end{dembar}


\begin{exbar}
\begin{example}
	\begin{center} 
		$e^z=\sum_{k=0}^{\infty}\frac{z^k}{k!}$, esponenziale complesso
	\end{center}
	
	Calcoliamo il raggio di convergenza
	
	$$\lim_{k\rightarrow+\infty}\sqrt[k]{\frac{1}{k!}}=\lim_{k\rightarrow+\infty}\frac{1}{\sqrt[k]{k!}}$$
	 
	$$\lim_{k\rightarrow+\infty}\frac{\frac{1}{(k+1)!}}{\frac{1}{k!}}=\lim_{k\rightarrow+\infty}\frac{1}{k+1}=0$$
	
	$$\Rightarrow R=+\infty$$
	
	$\Rightarrow$ la serie converge assolutamente in $\mathbb{C}$ e totalmente in $B_c(0) \,\, \forall \delta>0$.
\end{example}
\end{exbar}


\begin{exbar}
\begin{example}
	\begin{align*} 
		&\sin x =\sum_{k=0}^{\infty}(-1)^k\frac{x^{2k+1}}{(2k+1)!}
		\\
		&\cos x =\sum_{k=0}^{\infty}(-1)^k\frac{x^{2k}}{(2k)!}
	\end{align*}
	
	Studio la serie
	
	$$\sum_{k=0}^{\infty}(-1)^k\frac{(x^2)^{k}}{(2k+1)!}
	\uppercomment{=}{\myarrow[90]}{y=x^2}
	\lowercomment{\sum_{k=0}^{\infty}(-1)^k\frac{y^k}{(2k+1)!}}{\text{serie di potenze}}{\text{con } a_k=\frac{(-1)^k}{(2k+1)!}}$$ 
	
	Calcoliamo il raggio di convergenza 
	
	$$\lim_{k\rightarrow +\infty} \left|\frac{a_{k+1}}{a_k}\right| =\lim_{k\rightarrow+\infty}\frac{\frac{1}{(2k+3)!}}{\frac{1}{(2k+1)!}}=0$$
	
	$$\Rightarrow R=+\infty$$.
	
	Definiamo 
	
	$$\sin(z)=\sum_{k=0}^{\infty}(-1)^k\frac{z^{2k+1}}{(2k+1)!},\,\,\,\, z \in \mathbb{C}$$
	
	Il raggio di convergenze è sempre $R=+\infty$, e quindi la serie converge assolutamente in $\mathbb{C}$ e totalmente in $B_\delta(0)\forall \delta >0$. 
	
	La stessa cosa si può fare con la funzione coseno e definire 
	
	$$\cos z=\sum_{k=0}^{\infty}(-1)^k\frac{z^{2k}}{(2k)!},\,\,\,\, z \in \mathbb{C}$$
	
	Sia $x \in \mathbb{R}$
	
	\begin{align*} 
		\sin(ix)
		&=\sum_{k=0}^{\infty}(-1)^k\frac{(ix)^{2k+1}}{(2k+1)!}=\sum_{k=0}^{\infty}(-1)^k\frac{(i)^{2k+1}(x)^{2k+1}}{(2k+1)!} \lowercomment{=}{i^{2k+1=i^{2k}\cdot i}=}{=(-1)^k\cdot i}
		\\
		&=\sum_{k=0}^{\infty}(-1)^k(-1)^ki\frac{x^{2k+1}}{(2k+1)!}=i\sum_{k=0}^{\infty}\frac{x^{2k+1}}{(2k+1)!}=
		\\
		&=i\sinh x
	\end{align*}
	\begin{align*}
		\cos(ix)
		&=\sum_{k=0}^{\infty}(-1)^k \frac{(ix)^{2k}}{(2k)!}=\sum_{k=0}^{\infty}(-1)^k i^{2k} \frac{x^{2k}}{(2k)!}=\sum_{k=0}^{\infty}\frac{x^{2k}}{(2k)!}=
		\\
		&=\cosh x
	\end{align*}
	
	Sfruttando le serie scritte sopra si dimostra che 
	
	$$e^{x+iy}=e^x(\cos y + i \sin y) \,\,\,\, \forall x,y\in \mathbb{R}$$ 
	
	ed inoltre che 
	
	$$e^{z_1+z_2}=e^{z_1}e^{z_2},\,\,\, z_1,z_2 \in \mathbb{C}$$
\end{example}
\end{exbar}


\textbf{Osservazione:}


$z \in \mathbb{C}, z=x+iy$

$$e^{z+2\pi i}=e^{x+(y+2\pi)i}=e^x(\cos(y+2\pi)+i\sin(y+2\pi))=e^x(\cos y + i \sin y) =e^{x+iy}=e^z$$

$z \mapsto e^z$ è funzione periodica di periodo $2\pi i$.

$$e^z=e^x(\cos y+i \sin y)\neq 0\,\,\, \forall \,\, z \in \mathbb{C}$$

Fissiamo $w \in \mathbb{C}$ e proviamo a risolvere in $z =x+iy$ l'equazione $e^z=w$, $w=\rho (\cos \omega + i \sin \omega)$, $\rho =|\omega|$, $e^x(\cos y+i \sin y)=|\omega|(\cos \omega+i\sin \omega)$

\begin{equation*}
	\begin{cases}
		e^x=|w| 
		\\
		y=\theta+2k\pi, & \exists k\in\mathbb{Z}
	\end{cases} 
	\Rightarrow 
	\begin{cases}
		e=\ln |w|
		\\
		y=\theta+2k\pi, & \exists k\in\mathbb{Z}
	\end{cases}
\end{equation*}

$z =\ln |w|+i(\arg w + 2k\pi) \exists k \in \mathbb{Z}$, dove $\arg w \in ]-\pi,\pi]$ è l'argomento principale di $w$. 

In particolare, l'immagine di $z \mapsto e^z$ è $\mathbb{C} \backslash \{0\}$.

E' possibile definire la funzione \textbf{logaritmo principale} di un numero complesso 
\begin{gather*}
	\Ln: \mathbb{C} \backslash\{0\}\rightarrow \mathbb{C}
	\\
	\Ln w=\ln|w|=i \arg w
	\\
	\Ln (-1)=\ln|-1|+i\arg(-1)=i\pi
	\\
	\Ln (i)=\ln |i|+i\arg(i)=i \frac{\pi}{2}
	\\
	\Ln(\sqrt{2}+i\sqrt{2})=\ln|\sqrt{2}+i\sqrt{2}|+i \arg(\sqrt{2}+i\sqrt{2})=\ln2+i\frac{\pi}{4}
\end{gather*}


\begin{exbar}
\begin{example}
	Studiare la convergenza delle seria serie di funzioni
	\begin{equation*}
		\sum_{k=0}^{\infty}\frac{k!+2^k}{(2k)!-(\arctan k)^{k+1}}(\overbrace{\arctan x}^{\color{blue}y})^k,\,\,\,\,\, x \in \mathbb{R}
	\end{equation*}
	
	Studiamo la serie di potenze 
	
	$$\sum_{k=0}^{\infty} \underbrace{\frac{k!+2^k}{(2k)!-(\arctan k)^{k+1}}}_{\color{blue}a_k} y^k, \qquad y \in \mathbb{R}$$
	
	Calcoliamo il raggio di convergenza 
	\begin{gather*} 
		\lim_{k\rightarrow +\infty} \left| \frac{a_k+1}{a_k} \right| =
		\\
		=\lim_{k\rightarrow+\infty} 
		\frac{\uppercomment{(k+1)!+2^{k+1}}{}{\sim (k+1)!}}
		{\lowercomment{(2k+2)!-(\arctan(k+1))^{k+2}}{\sim (2k+1)!}{}}
		\frac{\uppercomment{(2k)!-(\arctan k)^{k+1}}{}{\sim (2k)!}}
		{\lowercomment{k!+2k}{\sim k!}{}}=
		\\
		\PdS \lim_{k \rightarrow+\infty} \frac{(k+1)!}{(2k+2)!}\cdot\frac{(2k)!}{k!} =\lim_{k\rightarrow +\infty}\frac{k+1}{(2k+2)(2k+1)}=0
	\end{gather*}
	
	$$ \Rightarrow R =+\infty$$
	
	La serie di potenze converge assolutamente $\forall y \in \mathbb{R}$ e totalmente in $]-\delta,\delta[$ $\forall \delta >0$. Poiché $\arctan x \in ]-\frac{\pi}{2},\frac{\pi}{2}[$ $\forall x \in \mathbb{R}$, la serie di partenza converge totalmente, e quindi uniformemente, in $\mathbb{R}$. 
\end{example}
\end{exbar}


\begin{exbar}
\begin{example}
	Studiare la serie di potenze
	\begin{equation*}
		\sum_{k=1}^{\infty} \lowercomment{\frac{i^k}{k}}{a_k}{} z^k.    
	\end{equation*}
	Calcoliamo il raggio di convergenza 
	
	$$\lim_{k \rightarrow +\infty}\sqrt[k]{|a_k|}=\lim_{k\rightarrow+\infty} \sqrt[k]{|\frac{i^k}{k}|}=\lim_{k \rightarrow +\infty}\frac{1}{\sqrt[k]{k}}=1$$
	
	$$\Rightarrow R=1$$
	
	La serie converge assolutamente per $z \in B_1(0)$ e totalmente in $B_\delta (0)\forall 0 < \delta<1$, e non converge se $|z|>1$.
	
	Dobbiamo studiare la convergenza per $|z|= 1$, cioè se $z =e^{i\theta}$ $\exists \theta \in [0,2\pi]$. Bisogna studiare
	
	$$\sum_{k=1}^{\infty}\frac{i^k}{k}e^{ik\theta}=\sum_{k=1}^{\infty}\frac{i^k}{k}(e^{i\theta})^k=\sum_{k=1}^{\infty} \lowercomment{\frac{1}{k}}{\alpha_k}{} \lowercomment{(ie^{i\theta})^k}{\beta_k}{}$$
	
	$\{\alpha_k\}_{k \geq 1}$ è monotona decrescente, infinitesima. 
	
	Studiamo la successione delle ridotte di $\sum_{k=1}^{\infty}\beta_k $ che è la serie di partenza di ragione $ie^{i\theta}$. 
	
	Se $i e^{i\theta}=1$, cioè $\theta =\frac{3}{2}\pi$, la serie si scrive $\sum_{k=1}^{\infty}\frac{1}{k}$, che diverge. 
	
	Sia $i e^{i\theta}\neq 1$, $\theta \neq \frac{3}{2}\pi$.
	
	$$S_n=\sum_{k=1}^{n}(ie^{i\theta})^k=\frac{1-i^{n+1}e^{i(n+1)\theta}}{1-ie^{i\theta}}-1$$
	
	$$|S_n|\leq \frac{|1-i^{n+1}e^{i(n+1)\theta}|}{|1-ie^{i\theta}|}+1 \leq \frac{2}{|i -ie^{i\theta}|}+1$$ 
	
	$\Rightarrow$ la successione delle ridotte della serie $\sum_{k=1}^{\infty}(i e^{i\theta})^k$ è limitata per $\theta \neq \frac{3}{2}\pi$ $\Rightarrow$ la serie $\sum_{k=1}^{\infty}\frac{1}{k}(ie^{i\theta})^k$ converge per $\theta \neq \frac{3}{2}\pi$ grazie al criterio di Dirichlet.
	
	La serie di partenza $\sum_{k=1}^{\infty}\frac{i^k}{k}z^k$ converge semplicemente in 
	
	$$\{z \in \mathbb{C} \; \big| \; |z|\leq 1, z \neq -i\}$$
\end{example}
\end{exbar}


\subsection{Cenno alla funzioni olomorfe}
Deriviamo $\sum_{k=0}^{\infty}a_kz^k$ ottenendo $\sum_{k=1}^{\infty}ka_kz^{k-1}$.

$$Dz^k \big|_{z=z_0}=\lim_{z \rightarrow z_0} \frac{z^k-z_0^k}{z-z_0}
\lowercomment{=}{\myarrow[270]}{z^k-z_0^k = (z-z_0)(z^{k-1} + z_0z^{k-2} + z_0^2 z^{k-3} + \ldots + z_0^{k-1})} kz_0^{k-1}$$

Qual è il raggio di convergenza di $\sum_{k=1}^{\infty}ka_k z^{k-1}$?

$$\limsup_{k \rightarrow +\infty} \sqrt[k]{k|a_k|}
\lowercomment{=}{\myarrow[270]}{\sqrt[k]{k} \to 1}
\limsup_{k \rightarrow +\infty}\sqrt[k]{|a_k|}=\frac{1}{R}$$

dove $R$ è il raggio di convergenza di $\sum_{k=0}^{\infty}a_kz^k$.



Prendiamo una serie di potenze in $\mathbb{R}$
\begin{equation*}
	f(x)=\sum_{k=0}^{\infty}a_kx^k,\,\,\,\, x \in \mathbb{R}
\end{equation*}

Sia $R>0$ il suo raggio di convergenza, allora $\sum_{k=0}^{\infty}a_kx^{k-1}$ ha raggio di convergenza $R>0 \Rightarrow$ la serie delle derivate converge totalmente, e quindi uniformemente, in ogni intervallo del tipo $]-\delta,\delta[$, $0<\delta<R$ $\Rightarrow f$ è derivabile in $]-\delta,\delta[$ $\forall\,\, 0<\delta<R$, e quindi in $]-R,R[$ e 
\begin{equation*}
	f'(x)=\sum_{k=0}^{\infty}ka_kx^{k-1}.
\end{equation*}

Deriviamo un'altra volta, e si ottiene che la serie derivata
\begin{equation*}
	\sum_{k=0}^{\infty} k(k-1)a_kx^{k-2}
\end{equation*}

ha raggio di convergenza $R$ e 

$$(f')'(x)=f''(x)=\sum_{k=0}^{\infty} k(k-1)a_kx^{k-2}$$

e quindi $f$ è derivabile due volte. Iterando il procedimento si ottiene che $f$ è di classe $C^\infty$ in $]-R,R[$.


\begin{definition}
	$A \subseteq \mathbb{C}$ aperto, $f:A\rightarrow \mathbb{C}$, $z_0\in A$. $f$ si dice derivabile (in senso complesso) in $z_0$ se $\exists$ finito 
	\begin{equation*}
		\lim_{z \rightarrow z_0} \frac{f(z)-f(z_0)}{z-z_0}=f'(z_0).
	\end{equation*}
	
	In particolare, $f$ si dice \textbf{olomorfa} in $A$ se è derivabile in senso complesso in ogni punto di $A$.
\end{definition}


\begin{exbar}
\begin{example}
	$f(z)=\overline{z}$, $f(x+iy)=x-iy$
	
	$$f'(0)=\lim_{z \rightarrow 0} \frac{\overline{z}-\overline{0}}{z-0}=\lim_{z\rightarrow0}\frac{\overline{z}}{z}$$
	
	Se $z$ è puramente immaginario, $z=iy$, 
	$$\frac{\overline{z}}{z}=\frac{-iy}{iy}=-1\xrightarrow{z \rightarrow 0}-1$$
	
	Se $z$ è puramente reale, $z=x$, 
	
	$$\frac{\overline{z}}{z}=1\xrightarrow{z \rightarrow 0}1$$
	
	$\Rightarrow \lim_{z\rightarrow0}\frac{\overline{z}}{z}$ non esiste, cioè $f$ non è derivabile in senso complesso. Se guardo $f$ come una funzione definita su $\mathbb{R}^2$,
	
	$$f(x,y)=(x,-y)=\begin{pmatrix}
		1 & 0 \\
		0 & -1 
	\end{pmatrix} \begin{pmatrix}
		x\\
		y
	\end{pmatrix}$$
	
	cioè $f$ è lineare, di classe $C^\infty$ come funzione di $(x,y)$.
	
	Si può dimostrare che se $f$ è olomorfa in un aperto $A\subseteq \mathbb{C}$, allora $\forall\,\, z_0 \in A$ $\exists\,\, R>0\mid f$ è somma di una serie di potenze in $B_R(z_0)$ di centro $z_0$, e quindi, come per le serie di potenze in $\mathbb{R}$, è derivabile in senso complesso infinite volte e vale
	\begin{equation*}
		f(z)=\sum_{k=0}^{\infty}\frac{f^{(k)}(z_0)}{k!}(z-z_0)^k
	\end{equation*}
	
	$\forall \,\, z \in B_R(z_0)$, cioè in $B_R(z_0)$, $f$ è somma della sua serie di Taylor di centro $z_0$.
	
	$f$ derivabile in senso complesso in $A \Rightarrow f$ è somma della sua serie di Taylor di centro $z_0$ in un intorno di $z_0$. 
\end{example}
\end{exbar}


\begin{attbar}
	\textbf{Osservazione:}
	
	Se $f:\mathbb{R}\rightarrow\mathbb{R}$ è di classe $C^\infty$, non è detto che sia somma della sua serie di Taylor.
	\begin{equation*}
		f(x)=\begin{cases}
			e^{-\frac{1}{x^2}} & \text{se }  x \neq 0 
			\\
			0 & \text{se } x =0
		\end{cases}
	\end{equation*}
	
	$f\in C^\infty(\mathbb{R})$, $f^{(k)}(0)=0$ $\forall\,\, k\in \mathbb{N}$ e quindi $f(x)\neq \sum_{k=0}^{\infty}\frac{\uppercomment{f^{(k)}(0)}{}{=0 }}{k!}x^k$, serie di Taylor di $f$ di centro zero.
\end{attbar}


\begin{exbar}
\begin{example}
	$e^z=\sum_{k=0}^{\infty}\frac{z^k}{k!}$, $z \in \mathbb{C}$.
	
	\begin{align*} 
		De^z&=\sum_{k=1}^{\infty}k\frac{z^{k-1}}{k!}=\sum_{k=0}^{\infty} k \frac{z^{k-1}}{k(k-1)!}=
		\\
		&=\sum_{k=0}^{\infty} \frac{z^{k-1}}{(k-1)!}\lowercomment{=}{j=k-1}{} \sum_{j=0}^{\infty} \frac{z^{j}}{j!}=e^z
	\end{align*}
	
	$$sin z= \sum_{k=0}^{\infty}(-1)^k\frac{z^{2k+1}}{(2k+1)!}$$
	
	\begin{align*} 
		D\sin z &= \sum_{k=0}^{\infty}(-1)^k(2k+1)\frac{z^{2k}}{(2k+1)!} = \sum_{k=0}^{\infty}(-1)^k(2k+1)\frac{z^{2k}}{(2k+1)(2k)!}
		\\
		&=\sum_{k=0}^{\infty}(-1)^k\frac{z^{2k}}{(2k)!}=\cos z
	\end{align*}
	
	$D\cos z=-\sin z$
	
	$\Ln z = \ln |z|+i \arg z$
	
	$\Ln $ è olomorfa in $\mathbb{C}\backslash\{z \in \mathbb{C} \; \big|$ $\Im(z)=0 \text{  e  } \Re(z) \leq 0 \}$, cioè è olomorfa in $\mathbb{C}$ privato della semiretta dei numeri reali non positivi 
	
	$$e^{\Ln z}=z \qquad \forall z \in \mathbb{C}\backslash\{z \in \mathbb{Z} \; \big| \; \Im(z)=0 \text{  e  } \Re(z) \leq 0 \}$$
	
	$D e^{ \Ln z}=Dz =1$
	
	$e^{\Ln z}D\Ln z=zD\Ln z$
	
	$$\Rightarrow D \Ln z =\frac{1}{z}$$
	
	\color{blue}{ \begin{center} 
			$\Ln: \mathbb{C}\backslash\{0\}\rightarrow\mathbb{C}$
		
			$\Ln z=\ln |z|=\ln |z|+i \arg z$
			
			Non è continua in $\{z \in \mathbb{C} \; \big| \Im(z)=0$ e $\Re(z)\leq 0\}$, $\arg(z)\in]-\pi,\pi]$
		\end{center}
	}
	
	\segnaposto %pag 305
\end{example}
\end{exbar}
