\section{Calcolo differenziale in $\mathbb{R}^n$}

\begin{definition}
	$A \subseteq \mathbb{R}^n$, $\overline{x_0}\in \mathbb{R}^n$ punto di accumulazione per $A$, $f:A\rightarrow \mathbb{R}$
	
	$$ \lim_{\overline{x}\rightarrow\overline{x_0} }f(0)=\ell \in \mathbb{R}$$
	
	$\Leftrightarrow \forall$ intorno $V$ di $l\,\, \exists\,\,$ un intorno $U$ di $\overline{x_0}\mid$ $\overline{x}\in U \cap A$, $\overline{x}\neq \overline{x_0}$ $\Rightarrow f(\overline{x}\in V)$ 
	
	$\Leftrightarrow \forall \epsilon >0$ $\exists\,\, \delta >0 \mid$ $0 < \|\overline{x}-\overline{x_0} \| <\delta$, $\overline{x}\in A\Rightarrow |f(\overline{x})-l|< \epsilon$
	
	$\Leftrightarrow |f(\overline{x})-l|\xrightarrow{\overline{x}\rightarrow\overline{x_0}}0$
\end{definition}


\begin{exbar}
	$$\lim_{(x,y)\rightarrow(0,0)}\frac{\sin(xy)}{xy}, \qquad f(x,y)=\frac{\sin(xy)}{xy}, \qquad xy\neq 0$$
	
	$$(x,y)\rightarrow (0,0)\Rightarrow xy \rightarrow 0$$
	
	$$\sin(xy)=xy+o(xy)$$
	
	$$\lim_{(x,y)\rightarrow(0,0)} \frac{\sin(xy)}{xy}=\lim_{(x,y)\rightarrow(0,0)}\frac{xy+o(xy)}{xy} \PdS \lim_{(x,y)\rightarrow(0,0)}\frac{xy}{xy}=1$$.
\end{exbar}


\begin{exbar}
	Calcoliamo, se esiste
	\begin{align*}
		\lim_{(x,y)\rightarrow(0,0)}x\frac{\sin(xy)}{x^2+y^2}
		&=\lim_{(x,y)\rightarrow(0,0)}x \frac{xy+o(xy)}{x^2+y^2}=\lim_{(x,y)\rightarrow(0,0)}\frac{x^2y+o(x^2y)}{x^2+y^2}=
		\\
		&\PdS \lim_{(x,y)\rightarrow(0,0)}\frac{x^2y}{x^2+y^2}
		\\
		\lim_{(x,y)\rightarrow(0,0)} \frac{x^2y}{x^2+y^2} &=0 
	\end{align*} 
	
	$$0 \leq \left|\frac{x^2y}{x^2+y^2} \right| = \uppercomment{\frac{x^2}{x^2+y^2}}{}{\leq 1}|y|\leq |y| \rightarrow 0$$.
\end{exbar}


\begin{exbar}
\begin{example}
	Calcolare, se esiste,
	\begin{equation*}
		\lim_{(x,y)\rightarrow (0,0)}\frac{\ln(1+\uppercomment{xy}{0}{\myarrow[90]})-\sin(\uppercomment{xy}{0}{\myarrow[90]})}{2x^2+y^2}
	\end{equation*} 
	\begin{align*} 
		\ln(1+t) &=t-\frac{t^2}{2}+o(t^2) \text{ per } t \rightarrow 0
		\\
		\sin(t) &=t-\frac{t^3}{6}+o(t^4) \text{ per } t \rightarrow 0
		\\
		\ln(1+xy) &=xy-\frac{x^2y^2}{2}+o(x^2y^2) \text{ per } (x,y)\rightarrow (0,0)
		\\
		\sin(xy) &=xy-\frac{x^3y^3}{6}+o(x^4y^4) \text{ per } (x,y)\rightarrow (0,0)
		\\
		\ln(1+xy)-\sin(xy) &=-\frac{x^2y^2}{2}+\frac{x^3y^3}{6}+o(x^2y^2)=
		\\
		&=-\frac{x^2y^2}{2}+o(x^2y^2) \text{ per } (x,y)\rightarrow (0,0)
	\end{align*}
	
	$$\lim_{(x,y)\rightarrow(0,0)}\frac{\ln(1+xy)-\sin(xy)}{2x^2+y^2} \PdS \lim_{(x,y)\rightarrow(0,0)} -\frac{1}{2}\frac{x^2y^2}{2x^2+y^2}=0$$
	
	\begin{itemize}
		\item $\bigg| -\frac{1}{2} \frac{x^2y^2}{\lowercomment{2x^2+y^2}{\geq 2x^2}{}} \bigg|\leq \frac{1}{2}\frac{x^2y^2}{2x^2}=\frac{1}{4}y^2\rightarrow 0$
		
		\item $\bigg|-\frac{1}{2}\frac{x^2y^2}{2x^2+y^2} \bigg| =\frac{1}{2}x^2 \uppercomment{\frac{y^2}{2x^2+y^2}}{}{\leq1} \leq \frac{1}{2}x^2\rightarrow 0$
		
		\item $|xy|\leq \frac{1}{2}(x^2+y^2)$
		
		$$0 \leq (|x|-|y|)^2=x^2-2|x||y|+y^2$$
	
		$$2|xy|\leq x^2+y^2$$
		
		$$ \bigg| \frac{1}{2}\frac{x^2y^2}{\lowercomment{2x^2+y^2}{\geq x^2+y^2}{}} \bigg|\leq \frac{1}{2}|xy| \uppercomment{\frac{|xy|}{x^2+y^2}}{}{\leq \frac{1}{2}}\leq \frac{1}{4}|xy|\rightarrow 0$$
	\end{itemize}
\end{example}	
\end{exbar}


\begin{exbar}
\begin{example}
	\begin{equation*}
		\lim_{(x,y)\rightarrow(0,0)}\frac{e^{xy}-\cos(xy)}{x^2-x^4+|y|}
	\end{equation*}
	\begin{align*} 
		e^{xy} &=1+xy+o(xy) \text{ per } (x,y)\rightarrow(0,0)
		\\
		\cos(xy) &=1-\frac{x^2y^2}{2}+o(x^2y^2) \text{ per } (x,y)\rightarrow(0,0)
		\\
		e^{xy}-\cos(xy) &=xy+o(xy)
	\end{align*}
	
	$$\lim_{(x,y)\rightarrow(0,0)}\frac{e^{xy}-\cos(xy)}{x^2-x^4-|y|} \PdS \lim_{(x,y)\rightarrow(0,0)}\frac{xy}{x^2-\lowercomment{x^4}{\color{blue} o(x^2+|y|)}{\color{red}??}+|y|}$$
	
	\begin{center} 
		{\color{blue}
			$x^2-x^4+|y|\geq x^2-x^4>0$ in un intorno di $x=0$
			
			$\left|\frac{xy}{x^2-x^4+|y|}\right|\leq \frac{|xy|}{x^2-x^4}$
			
			$\lim_{(x,y)\rightarrow(0,0)}\frac{|xy|}{x^2-x^4}=o(x^2)=\lim_{(x,y)\rightarrow(0,0)} \frac{|x|}{x^2}|y|=\lim_{(x,y)\rightarrow(0,0)} \left|\frac{x}{y}\right|$ } {\color{red}???}
	\end{center}
	
	$x^2-x^4 \geq 0$ definitivamente per $(x,y)\rightarrow(0,0)$
	
	$x^2-x^4+|y|\geq |y|$ definitivamente per $(x,y)\rightarrow(0,0)$
	
	$$ \left| \frac{xy}{x^2-x^4+|y|} \right|\leq \frac{|xy|}{|y|}=|x|\xrightarrow{(x,y)\rightarrow(0,0)} 0$$
	
	e quindi il limite esiste e vale zero.
\end{example}
\end{exbar}


\subsection{Limiti all'infinito}
Avevamo introdotto il simbolo  $\infty$ in $\mathbb{R}^n$. Gli intorni di infinito sono complementari di palle di centro l'origine.

Dire che $\overline{x} \rightarrow \infty$ $\Leftrightarrow \|\overline{x}\|\rightarrow +\infty$, $f:\dom f\rightarrow \R$, $\infty$ punto di accumulazione per $\dom f \subseteq \R^n$

$$\lim_{\overline{x}\rightarrow \infty} f(\overline{x})=l\in \R \Leftrightarrow$$

$\forall \,\, \epsilon >0$ $\exists\,\, M >0 \,\, \mid$ $\|\overline{x}\|>M,$ $\overline{x}\in \dom f$ $\Rightarrow |f(\overline{x})-\ell|<\epsilon$.

{\color{blue}($\forall$ intorno $V$ di $l$ $\exists$ un intorno $U$ di $\infty \mid \overline{x}\in U \cap \dom f\Rightarrow f(\overline{x})\in V$.)}


\begin{exbar}
	\begin{equation*}
		\lim_{(x,y)\rightarrow\infty} \arctan( \uppercomment{x^2+|y|}{}{+\infty})=\frac{\pi}{2}
	\end{equation*}
	\begin{equation*}
		\lim_{(x,y)\rightarrow\infty} \arctan(x^2+y)= \nexists
	\end{equation*}
	
	\image{calcolo_differenziale/pag312_1.png} %pag 312_1
	
	Se $y=-x^2 \Rightarrow \arctan (y+x^2) \mid_{y=-x^2} =0$ 
	
	\image{calcolo_differenziale/pag312_2.png} % pag 312_2
\end{exbar}


\begin{exbar}
	$$p(x,y)=x^2+y$$
	 
	$$\lim_{(x,y)\rightarrow\infty}p(x,y) \,\,\nexists$$
	
	$p(x,y)\big|_{y+x^2=0}=0$
	
	$p(x,y)\big|_{y+x^2=1}=1$
	
	\image{calcolo_differenziale/pag313.png} %pag 313 
\end{exbar}


\begin{theorem} \textbf{sui limiti lungo restrizioni}
	
	$f:\dom f\rightarrow\R$, $\dom f \subseteq\R^n$, $\overline{x_0}\in \R^n\cup\{\infty\}$ punto di accumulazione per $\dom f$.
	
	Allora $\lim_{\overline{x}\rightarrow\overline{x_0}}f(\overline{x})=\ell$ $\Leftrightarrow \forall\,A \subseteq \dom f$ tale che $\overline{x_0}$ è punto di accumulazione per $A$ si ha che 
	\begin{equation*}
		\lim_{\overline{x}\rightarrow\overline{x_0}}f\big|_A(\overline{x})=\ell
	\end{equation*}
	
	{\color{blue}
		$f\big|_A:A\rightarrow \R$, $\overline{x}\mapsto f(\overline{x})$ restrizione di $f$ ad $A$
	
		$$\lim_{\genfrac{}{}{0pt}{}{\overline{x}\rightarrow\overline{x_0}}{\overline{x}\in A}} f\big|_A(\overline{x})=\ell $$
	 }
\end{theorem}


\begin{attbar}
	Solitamente il teorema viene utilizzato per dimostrare che un limite non esiste, ad esempio trovando due restrizioni con due limiti diversi, come fatto negli esempi precedenti.
\end{attbar}


\begin{exbar}
\begin{example}
	Calcolare, se esiste,
	\begin{equation*}
		\lim_{(x,y)\rightarrow(0,0)} \uppercomment{\frac{xy}{x^2+y^2}}{}{f(x,y)}
	\end{equation*}
	
	Consideriamo la restrizione di $f$ lungo una retta uscente dall'origine di equazione $y= mx$
	
	\begin{align*} 
		f(x,y)\big|_{y=mx}
		&=f(x,mx)=\frac{mx^x}{x^2+m^2x^2}=
		\\
		=\frac{m}{1+m^2}\xrightarrow{(x,y)\rightarrow(0,0)}\frac{m}{1+m^2}
	\end{align*} 
	
	{\centering $\Rightarrow f$ non ha limite in $(0,0)$. \par}
\end{example}
\end{exbar}


\begin{exbar}
\begin{example}
	Provare che 
	\begin{equation*}
		\lim_{(x,y)\rightarrow(0,0)} \uppercomment{\frac{x^2y}{x^4+y^2}}{}{f(x,y)} \text{ non esiste}
	\end{equation*}
	\begin{align*}
		f(x,y)\big|_{y=mx}
		&=f(x,mx)=\frac{mx^3}{x^4+m^2x^2}=
		\\
		&=\frac{mx^3}{x^2(x^2+m^2)}=\frac{mx}{x^2+m^2}\xrightarrow{(x,y)\rightarrow(0,0)}0
	\end{align*}
	
	Le restrizioni di $f$ lungo le rette passanti per l'origine hanno limite zero in $(0,0)$.
	
	\textbf{Non posso dedurre che $f$ ha limite zero in $(0,0)$.}
	
	Se prendiamo la restrizione di $f$ lungo la parabola di equazione $y=x^2$ abbiamo
	\begin{equation*}
		f(x,y)\big|_{y=x^2}=f(x,x^2)=\frac{x^4}{x^4+x^4}\xrightarrow{(x,y)\rightarrow(0,0)} \frac{1}{2}
	\end{equation*}
	
	{\centering $\Rightarrow f$ non ha limite in $(0,0)$. \par}
\end{example}
\end{exbar}


\begin{exbar}
\begin{example}
	\begin{gather*}
		\lim_{(x,y)\rightarrow(0,0)} \uppercomment{\frac{xy}{y-x}}{}{f(x,y)}, \qquad y \neq x
		\\
		f(x,y)\big|_{y=mx}=f(x,mx)=\frac{mx^2}{x(m-1)}=\frac{mx}{m-1}\xrightarrow{(x,y)\rightarrow(0,0)}0
		\\
		f(x,y)\big|_{y=x+x^2}=f(x,x+x^2)= \frac{x(x+x^2)}{x^2}=\frac{x^2+x^3}{x^2}\xrightarrow{(x,y)\rightarrow(0,0)}1
	\end{gather*}
	
	{\centering $\Rightarrow f$ non ha limite in $(0,0)$. \par}
\end{example}
\end{exbar}


\subsection{Utilizzo delle coordinate polari}
	\image{calcolo_differenziale/pag316.png} %pag 316 

$(x_0,y_0)\in \R^2$. Coordinate polari di centro $(x_0,y_0)$ 
\begin{gather*}
	\begin{cases}
		x=x_0+\rho \cos(\theta)
		\\
		y=y_0+\rho \sin(\theta)
	\end{cases}
	\\
	\lim_{(x,y)\rightarrow(x_0,y_0)}f(x,y)=\ell\in \R 
\end{gather*}
\begin{gather*}
	\forall\,\, \epsilon >0 \,\, \exists \,\, \delta >0 \; \mid \; 0 < \overbrace{\|(x,y)-(x_0,y_0)\|}^{{\color{blue} =\delta}} \leq \delta,
	\qquad
	(x,y)\in \dom f \Rightarrow |f(x,y)-\ell|< \epsilon
	\\
	\forall \,\, \epsilon >0 \,\, \exists \delta >0 \mid 0<\rho < \delta
	\qquad
	(x_0+\rho \cos(\theta), y_0+\rho \sin(\theta))\in \dom f
	\\
	\Rightarrow |f(x_0+\rho \cos(\theta), y_0+\rho \sin(\theta))-\ell|<\epsilon \qquad \forall\theta \in [0,2\pi[
	\\
	\sup_{\theta \in [0,2\pi[}|f(x_0+\rho \cos(\theta), y_0+\rho \sin(\theta))-\ell|\leq\epsilon
	\\
	\lim_{\rho\rightarrow0}\sup_{\theta \in [0,2\pi[}|f(x_0+\rho \cos(\theta), y_0+\rho \sin(\theta))-\ell|=0
\end{gather*}


\begin{theorem}
	$f:\dom f \rightarrow\R$, $\dom f \subseteq\R^2$, $(x_0,y_0)\in \R^2$ punto di accumulazione per $\dom f$. Supponiamo {\color{blue}(per semplicità)} che esista un intorno $U$ di $(x_0,y_0) \; \mid$ $U\backslash \{(x_0,y_0)\}\subseteq \dom f$
	\begin{enumerate}
		\item $\lim_{(x,y)\rightarrow(x_0,y_0)}f(x,y)=\ell\in\R \Leftrightarrow$ $\lim_{\rho \rightarrow 0 } \sup_{\theta \in [0,2\pi[}|f(x_0+\rho\cos(\theta),y_0+\rho\sin(\theta))-\ell|=0$ 
		
		\item $\lim_{(x,y)\rightarrow(x_0,y_0)}f(x,y)=+\infty \Leftrightarrow$ $\lim_{\rho\rightarrow 0}\inf_{\theta \in[0,2\pi[}|f(x_0+\rho \cos(\theta), y_0+\rho \sin(\theta))|=+\infty$
		
		\item $\lim_{(x,y)\rightarrow(x_0,y_0)}f(x,y)=-\infty\Leftrightarrow$  $\lim_{\rho\rightarrow 0}\sup_{\theta \in[0,2\pi[}|f(x_0+\rho \cos(\theta), y_0+\rho \sin(\theta))|=-\infty$
	\end{enumerate}
\end{theorem}


\begin{exbar}
\begin{example}
	\begin{equation*}
		\lim_{(x,y)\rightarrow(0,0)} \uppercomment{\frac{5x^3+xy^2}{x^2+y^2}}{}{f(x,y)}
	\end{equation*}
	\begin{enumerate}
		\item Indovino il candidato valore del limite
		\begin{align*} 
			f(\rho\cos(\theta),\rho\sin(\theta)))
			&=\frac{5\rho^3\cos^2\theta+\rho^3\cos\theta\sin^2\theta}{\rho^2}=
			\\
			&=\rho[5\cos^2\theta+\cos\theta\sin^2\theta]\xrightarrow{\rho\rightarrow0}0 \qquad \forall \theta
		\end{align*}

		Il candidato limite è $\ell=0$

		\item Dimostro che effettivamente $f$ ha limite $l$ in $(0,0)$.
		
		Devo far vedere che 
		
		$$\sup_{\theta \in [0,2\pi[} |f(\rho\cos(\theta),\rho\sin(\theta))-\ell|=\sup_{\theta\in[0,2\pi[} |\rho(5\cos^3(\theta)+\cos(\theta)\sin^2(\theta))|\xrightarrow{\rho\rightarrow0}0$$
		
		$$|\rho(5\cos^3(\theta)+\cos(\theta)\sin^2(\theta))|\leq \rho(|5\cos^3\theta|+|\cos\theta\sin^2\theta|)\leq 6\rho$$
		
		$$\sup_{\theta\in[0,2\pi[}|\rho(5\cos^3\theta+\cos\theta\sin^2\theta)|\leq 6\rho \xrightarrow{\rho\rightarrow0}0$$
		
		$$\Rightarrow \lim_{(x,y)\rightarrow(0,0)\frac{5x^3+xy^2}{x^2+y^2}}=0$$
	\end{enumerate}
\end{example}
\end{exbar}


\begin{exbar}
\begin{example}
	Calcolare $\lim_{(x,y)\rightarrow(0,0)} \uppercomment{\frac{|x|^\alpha y}{x^2+2y^2}}{}{f(x,y)}$ al variare di $\alpha >0$.
	\begin{enumerate}
		\item Indovino il candidato valore limite
		
		\begin{align*}
			f(\rho\cos\theta,\rho\sin\theta)
			&=\rho^{1+\alpha}\frac{|\cos\theta|^\alpha\sin\theta}{\rho^2(\cos^2\theta+2\sin^2\theta)}=\rho^{\alpha-1}\frac{|\cos\theta|^\alpha \sin\theta}{\cos^2\theta+2\sin^2\theta}\rightarrow
			\\
			&\xrightarrow{\rho\rightarrow0}
			\begin{cases}
				0 &\text{  se  } \alpha > 1 
				\\
				\frac{|\cos\theta|^\alpha \sin \theta}{\cos^2\theta+2\sin^2\theta} &\text{  se  } \alpha=1  {\color{teal}*}
				\\
				\infty &\text{  se  } \alpha < 1 \text{  e   } \cos\theta \sin\theta \neq 0  {\color{blue}*}
				\\
				0 &\text{  se  } \alpha < 1  \text{  e   } \cos\theta \sin\theta = 0  {\color{blue}*}
			\end{cases}
		\end{align*}
		
		{\color{teal} Lungo semirette uscenti da $(0,0)$ ho limiti diversi perché il limite dipende da $\theta \Rightarrow$ non esiste.}
		
		{\color{blue} Ho limiti diversi a seconda che tenda a $(0,0)$ muovendomi lungo gli assi cartesiani o fuori da essi $\Rightarrow$ il limite non esiste.}
		
		Il limite può esistere solo per $\alpha > 1$.
		
		\item Proviamo a dimostrare che il limite esiste e fa zero per $\alpha >1$.
		
		$$\sup_{\theta \in [0,2\pi[} \left| f(\rho\cos\theta,\rho\sin\theta)-\ell \right| = \sup_{\theta \in [0,2\pi[} \left|\rho^{\alpha-1}\frac{|\cos \theta|^\alpha \sin\theta}{\cos^2\theta+2\sin^2\theta}-0 \right| {\color{blue} \xrightarrow[\text{per } \rho \to 0]{?}0}$$
		
		$$\bigg| \rho^{\alpha-1}\frac{ \uppercomment{|\cos \theta|^\alpha \sin\theta}{}{\leq 1}}{\cos^2\theta+2\sin^2\theta} \bigg| \lowercomment{\leq}{\cos^2\theta+2\sin^2\theta=\cos^2\theta+\sin^2\theta+\sin^2\theta=}{1+\sin^2\theta \geq 1} \rho^{\alpha-1}\frac{1}{1}=\rho^{\alpha-1}$$
		
		$$\sup_{\theta \in [0,2\pi[} \left| \rho^{\alpha-1}\frac{|\cos\theta|^\alpha\sin\theta}{\cos^2\theta+2\sin^2\theta} \right| \leq \rho^{\alpha-1}\xrightarrow{\rho\rightarrow0}0$$
		
		$\Rightarrow \lim_{(x,y)\rightarrow(0,0)}\frac{|x|^\alpha y}{x^2+2y^2}$ esiste $\Leftrightarrow \alpha> 1$ e in tal caso vale zero. 
	\end{enumerate}
\end{example}
\end{exbar}


\begin{exbar}
\begin{example}
	$f:\R^2\rightarrow \R$ definita da 
	\begin{equation*}
		f(x,y)=
		\begin{cases}
			\frac{\pi-2\arccos(1-e^{-(x^2+y^4)})}{(x^2+y^2)^\alpha}&\text{  se  }(x,y)\neq (0,0)
			\\
			0&\text{  se  }(x,y)=(0,0)
		\end{cases} \qquad\qquad \alpha \in \R.
	\end{equation*}
	
	Trovare tutti e soli i valori di $\alpha$ per cui $f$ è continua. In $\R^2 \backslash\{(0,0)\}$ $f$ è continua perché rapporto di funzioni continue con denominatore non nullo.
	
	L'esercizio si riduce a trovare tutti e soli i valori di $\alpha \in \R \mid$
	
	$$\lim_{(x,y)\rightarrow(0,0)}f(x,y)=0=f(0,0)$$
	
	Calcoliamo
	\begin{equation*}
		\lim_{(x,y)\rightarrow(0,0)}  \frac{\pi-2\arccos(\uppercomment{1-e^{-(x^2+y^4)}}{}{0 \text{ per } (x,y)\to(0,0)} )}{(x^2+y^2)^\alpha}
	\end{equation*}
	
	al variare di $\alpha \in \R$.
	
	\begin{align*}
		\arccos t &=\arccos(0)+\arccos'(0)t+o(t)=\frac{\pi}{2}-t+o(t)
		\\
		\arccos(1-e^{-(x^2+y^4)}) &=\frac{\pi}{2}-(1-e^{-(x^2+y^4)})+o(1-e^{-(x^2+y^4)})
		\\
		e^{-(x^2+y^4)} &=1-(x^2+y^4)+o(x^2+y^4)
		\\
		\arccos(1-e^{-(x^2+y^4)}) &=\frac{\pi}{2}-(x^2+y^4)+o(x^2+y^4)
		\\
		\pi-2\arccos(1-e^{-(x^2+y^4)}) &=2(x^2+y^4)+o(x^2+y^4) 
		\\
		\lim_{(x,y)\rightarrow(0,0)}f(x,y) &\PdS \lim_{(x,y)\rightarrow(0,0)} \lowercomment{2\frac{x^2+y^4}{(x^2+y^2)^\alpha}}{}{\phi(x,y)}
	\end{align*}
	
	\begin{enumerate}
		\item Individuiamo il candidato limite di $\phi$ passando in coordinate polari.
		
		$$\phi(\rho\cos(\theta),\rho\sin(\theta))=2\frac{\rho^2(\cos^2(\theta)+\rho^2\sin^4(\theta))}{\rho^{2\alpha}}=2\rho^{2(1-\alpha)}(\cos^2\theta+\rho^2\sin^4\theta)$$
		
		\begin{equation*}
			\lim_{\rho\rightarrow0}\phi(\rho\cos(\theta),\rho\sin(\theta))=
			\begin{cases}
				0&\text{ se }\alpha<1
				\\
				2\cos^2\theta&\text{ se }\alpha=1
				\\
				\infty&\text{ se }\alpha>1\text{ e }\cos^2(\theta)\neq 0
				\\
				0,1\text{ o }\infty\text{ a seconda dei casi }&\text{ se }\cos^2(\theta)=0
			\end{cases}
		\end{equation*}
		
		$\Rightarrow$ il limite può esistere solo per $\alpha< 1$ e in tal caso vale $0$.
		
		\item Cerchiamo di dimostrare che il limite vale zero stimando
		
		$\color{red} \alpha < 1$
		
		$$\sup_{\theta \in[0,2\pi[} |\phi(\rho\cos(\theta),\rho\sin(\theta))-\uppercomment{\ell}{}{=0}| = \sup_{\theta \in[0,2\pi[}|2\rho^{2(1-\alpha)}(\cos^2(\theta)+\rho^2\sin^4(\theta))|$$
		
		$$|2\rho^{2(1-\alpha)}(\cos^2(\theta)+\rho^2\sin^4(\theta))| \leq \underbrace{2\rho^{2(1-\alpha)}(1-\rho^2)}_{{\color{blue}\text{che non dipende da } \theta}}$$ 
		
		{\centering $\sup_{\theta \in[0,2\pi[}|2\rho^{2(1-\alpha)}(\cos^2(\theta)+\rho^2\sin^4(\theta))|\leq2\rho^{2(1-\alpha)}(1-\rho^2)\xrightarrow{\rho \rightarrow 0} 0 $ se $\alpha <1$ \par}

		$\Rightarrow \lim_{(x,y)\rightarrow(0,0)}f(x,y)=0$, e quindi $f$ è continua in $(0,0)$ $\Leftrightarrow \alpha<1$.
	\end{enumerate}
\end{example}
\end{exbar}


\begin{exbar}
\begin{example}
	Sia 
	\begin{equation*}
		\Gamma=\{(x,y,z)\in\R^3\mid x^2+y^2-z^2\leq 1, x+y+2z=0\} 
	\end{equation*}
	
	e sia $f:\R^3\rightarrow \R $ definita da 
	\begin{equation*}
		f(x,y,z)=
		\begin{cases}
			\frac{1-\cos(xyz)}{x^2y^2z^2}&\text{ se }xyz\neq 0\\
			\frac{1}{2}&\text{ se }xyz= 0
		\end{cases}
	\end{equation*}
	
	Provare che $f$ ha massimo e minimo in $\Gamma$.
	
	Se dimostriamo che $f$ è continua e $\Gamma$ è compatto, il teorema di Weierstrass permette di concludere.
	
	Fuori dai piani coordinati $f$ è continua perché rapporto di funzioni continue con denominatore non nullo.
	
	Sia $(x_0,y_0,z_0)$ un punto di un piano coordinato, cosicché $x_0y_0z_0=0$, e calcoliamo il
	
	$$\lim_{(x,y,z)\rightarrow(x_0,y_0,z_0)}f(x,y,z)$$
	
	Se $(x,y,z)\rightarrow(x_0,y_0,z_0)$, allora $xyz\rightarrow0$
	
	$$\cos(xyz)=1-\frac{x^2y^2z^2}{2}+o(x^2y^2z^2)$$
	
	$$\lim_{(x,y,z)\rightarrow(x_0,y_0,z_0)} \PdS \lim_{(x,y,z)\rightarrow(x_0,y_0,z_0)} \frac{\frac{x^2y^2z^2}{2}}{x^2y^2z^2}=\frac{1}{2}=f(x_0,y_0,z_0)$$
	
	$\Rightarrow f$ è continua anche in $(x_0,y_0,z_0)$.
	
	Proviamo a dimostrare che $\Gamma$ è compatto, cioè chiuso e limitato.
	\begin{align*} 
		\Gamma &=\{(x,y,z)\in\R^3\mid x^2+y^2-z^2\leq 1, x+y+2z=0\}
		\\
		\Gamma_1 &=\{(x,y,z)\in\R^3\mid x^2+y^2-z^2\leq 1\}
		\\
		\Gamma_2 &=\{(x,y,z)\in\R^3\mid x+y+2z=0\}
	\end{align*}
	\begin{center} 
		$\Gamma=\Gamma_1\cap\Gamma_2$
	
		$\phi_1(x,y,z)=x^2+y^2-z^2$, che è continua 
		
		$\Gamma_1=\{(x,y,z)\in\R^3\mid \phi_1(x,y,z)\leq 1\} =\phi_1^{-1}( \underbrace{]-\infty,1]}_{\color{blue} \text{chiuso}})$
	\end{center}
	
	$\Rightarrow \Gamma_1$ è anti-immagine di un chiuso tramite una funzione continua $\Rightarrow$ è chiuso.
	\begin{center}	
		$\phi_2(x,y,z)=x+y+2z$ è continua
	
		$\Gamma_2=\{(x,y,z)\in\R^3\mid \phi_2(x,y,z)= 0\}=\phi_2^{-1}(\underbrace{\{0\}}_{{\color{blue} \text{chiuso}}})$
	\end{center}
	
	$\Rightarrow\Gamma_2$ è anti-immagine di un chiuso tramite una funzione continua $\Rightarrow$ è chiuso $\Rightarrow$ siccome $\Gamma$ è intersezione di chiusi, è chiuso.
	
	Dimostriamo che $\Gamma $ è limitato.
	
	Devo far vedere che $\exists C>0 \;\mid$ $|x|,|y|,|z|\leq C$ $\forall (x,y,z)\in \Gamma$ oppure che $\exists C>0 \; \mid$ $\|(x,y,z)\|\leq C$ $\forall(x,y,z)\in\Gamma$.
	
	$$x^2+y^2-z^2\leq 1, \qquad z=-\frac{x+y}{2}$$
	
	$$x^2+y^2\leq 1+\left(\frac{x+y}{2}\right)^2=1+\frac{1}{4}(x^2+y^2+2xy)$$
	
	$$\frac{3}{4}(x^2+y^2)\leq 1+\frac{1}{2}xy\leq 1+\frac{1}{4}(x^2+y^2)$$
	
	$$\frac{1}{2}(x^2+y^2)\leq 1\Rightarrow x^2+y^2\leq 2\Rightarrow |x|,|y|\leq \sqrt{2}$$
	
	$$z=-\frac{x+y}{2}\Rightarrow |z|=\frac{|x+y|}{2}\leq \frac{|x|+|y|}{2}\leq \sqrt{2}$$
	
	Prese $C=\sqrt{2}$, si ha $|x|,|y|,|z|\leq C \; \forall(x,y,z)\in \Gamma$
	
	{\centering $\Rightarrow\Gamma$ è limitato. \par}
\end{example}
\end{exbar}


\subsection{Derivate parziali e direzionali}
$f:\R\rightarrow\R$, $x_0\in\R$
\begin{equation*}
	\lim_{x \rightarrow x_0}\frac{f(x)-f(x_0)}{x-x_0}=f'(x_0)
\end{equation*}

$f(x)=f(x_0)+f'(x_0)(x-x_0)+o(x-x_0)$

$f:\R^2\rightarrow\R$

\image{calcolo_differenziale/pag329.png} %pag 329

\begin{itemize}
	\item $f:\dom f\rightarrow\R$, $domf \subseteq \R^n$
	\item $\overline{x_0}\in \dom f$, punto interno
	\item $\overline{v}\in \R^n$ versore
\end{itemize}

Considero la restrizione di $f$ lungo la retta passante per $\overline{x_0}$ e parallela a $\overline{v}$

$$\phi_{\overline{v}}(t)=f(\overline{x_0}+t\overline{v})$$

ben definita in un intorno di $t=0$ perché $\overline{x_0}$ è interno a $\dom f$.


\begin{definition}
	Se $\phi_{\overline{v}}$ è derivabile in $t=0$ allora 
	\begin{equation*}
		\phi_{\overline{v}}'(0)=\lim_{t\rightarrow0}\frac{\phi_{\overline{v}}(t)-\phi_{\overline{v}}(0)}{t}=\lim_{t\rightarrow 0}\frac{f(\overline{x_0}+t\overline{v})-f(\overline{x_0})}{t}=D_{\overline{v}}f(x_0)
	\end{equation*}
	
	si dice derivata di $f$ in $\overline{x_0}$ nella direzione $\overline{v}$ e $f$ si dice derivabile in $\overline{x_0}$ nella direzione $\overline{v}$ o lungo $\overline{v}$.
\end{definition}


\begin{exbar}
	$$f(x,y)=e^{x+y}+xy \qquad \overline{v}=(\cos\alpha,\sin\alpha)$$
	
	Calcoliamo, se esiste, $D_{\overline{v}}f(x_0,y_0)$.
	
	\begin{align*} 
		\phi_{\overline{v}}(t) &=f((x_0,y_0)+t(\cos\alpha,\sin\alpha)) =f(x_0+t\cos\alpha,y_0+t\sin\alpha)=
		\\
		&=e^{x_0+y_0+t(\cos\alpha+\sin\alpha)}+(x_0+t\cos\alpha)(y_0+t\sin\alpha)
	\end{align*}
	
	$$\phi_{\overline{v}}'(0)=(\cos\alpha+\sin\alpha)e^{x_0+y_0}+y_0\cos\alpha+x_0\sin\alpha=D_{\overline{v}}f(x_0,y_0)$$
\end{exbar}


\begin{definition}
	Se $\overline{v}=\overline{e_k}$, $k-$esimo vettore della base canonica, $D_{\overline{e_k}}f(\overline{x_0})$ \textbf{si dice derivata parziale di $f$ in $x_0$ rispetto a $x_k$} {\color{teal}$(\overline{x}=(x_1,x_2,...,x_n))$} ed è indicata con uno dei simboli 
	
	$\partial_{x_k}f(\overline{x_0})$, $\frac{\partial f}{\partial x_k}(\overline{x_0})$, $D_{x_k}f(\overline{x_0})$, $f_{x_k}(\overline{x_0})$, $D_kf(\overline{x_0})$.
	
	Se esistono tutte $n$ le derivate parziali di $f$ in $\overline{x_0}$, $\partial_{x_1}f(\overline{x_0}),\partial_{x_2}f(\overline{x_0}),...,\partial_{x_n}f(\overline{x_0})$, \textbf{f si dice derivabile in $\overline{x_0}$} e il vettore 
	
	$$(\partial_{x_1}f(\overline{x_0}),\partial_{x_2}f(\overline{x_0}),...,\partial_{x_n}f(\overline{x_0}))$$
	
	si dice \textbf{gradiente di $f$ in $\overline{x_0}$} e si scrive $\nabla f(\overline{x_0})$ o grad$f(\overline{x_0})$.
\end{definition}


\begin{exbar}
	$$f:\R^2\rightarrow\R, \qquad (x_0,y_0)\in\R^2$$
	
	{\color{blue}
		$$\partial_xf(x_0,y_0)=\lim_{t\rightarrow0}\frac{f((x_0,y_0)+t\overline{e_1})-f(x_0,y_0)}{t}=\lim_{t\rightarrow0}\frac{f(x_0+t,y_0)-f(x_0,y_0)}{t}$$
	}
	
	\image{calcolo_differenziale/pag332.png} %pag 332
	
	{\color{teal}
		$$\partial_y f(x_0,y_0)=\lim_{t \rightarrow 0}\frac{f(x_0,y_0+t)-f(x_0,y_0)}{t}$$
	}

	$$\nabla f(x_0,y_0)=(\partial_xf(x_0,y_0),\partial_yf(x_0,y_0))$$
	
	
	$$f(x,y)=e^{x^2+y}+x^2y^3$$
	
	$\partial_xf(x,y)=2xe^{x^2+y}+2xy^3$
	
	$\partial_yf(x,y)=e^{x^2+y}+3x^2y^2$
	
	$$\nabla f(x,y)=(2xe^{x^2+y}+2xy^3,e^{x^2+y}+3x^2y^2)$$
\end{exbar}


\begin{attbar}
	$f,g:A\rightarrow\R$, $A\subseteq\R^n$ aperto e derivabile
	\begin{gather*}
		\partial_{x_k}(\alpha f+\beta g)(\overline{x_0})=\alpha\partial_{x_k}f(\overline{x_0})+\beta\partial_{x_k}g(\overline{x_0}) \text{  per  }   \alpha,\beta\in\R
		\\
		\partial_{x_k}(f\cdot g)(\overline{x_0})=g(\overline{x_0})\partial_{x_k}f(\overline{x_0})+f(\overline{x_0})\partial_{x_k}g(\overline{x_0})
		\\
		\partial_{x_k}\left(\frac{f}{g}\right)(\overline{x_0})=\frac{g(\overline{x_0})\partial_{x_k}f(\overline{x_0})-f(\overline{x_0})\partial_{x_k}g(\overline{x_0})}{(g(\overline{x_0}))^2} \text{  per  } g(\overline{x_0})\neq 0
	\end{gather*}
\end{attbar}


\begin{attbar}
	\textbf{Regola della catena:}
	
	$A\subseteq \R^n$ aperto, $g:A\rightarrow \R$, derivabile in $\overline{x_0}\in A$, $f:I\rightarrow\R$, $I$ intervallo, derivabile in $g(\overline{x_0})\in I$, allora $f\circ g$ è derivabile in $\overline{x_0}$ e 
	\begin{align*}
		\nabla(f \circ g)(\overline{x_0})&=f'(g(\overline{x_0}))\nabla g(\overline{x_0}),\\
		\partial_{x_k}(f\circ g)(\overline{x_0})&=f'(g(\overline{x_0}))\partial_{x_k}g(\overline{x_0})
	\end{align*}
	
	Se $f:\R\rightarrow\R$ è derivabile in $x_0$, allora
	\begin{enumerate}
		\item $f$ è continua in $x_0$
		\item posso definire la retta tangente al grafico di $f$ in $x_0$ come la retta di equazione
		\begin{equation*}
			y=f(x_0)+f'(x_0)(x-x_0).
		\end{equation*}
	\end{enumerate}
\end{attbar}


\begin{exbar}
\begin{example}
	\begin{equation*}
		f(x,y)=
		\begin{cases}
			\sin\frac{1}{xy}&\text{  se  }xy\neq0
			\\
			1& \text{  se  }xy=0
		\end{cases}
	\end{equation*}
	
	\begin{itemize} 
		\item $\partial_{x}f(0,0)$
		
		Per calcolarla, guardiamo la restrizione di $f$ lungo l'asse delle ascisse
		\begin{gather*}
			\phi(t)=f(t,0)=0\,\,\forall\,\,t\in\R
			\\
			\partial_xf(0,0)=\phi'(0)=0
		\end{gather*}
		
		\item $\partial_yf(0,0)$
		
		guardiamo $\psi(t)=f(0,t)$, restrizione di $f$ lungo l'asse delle ordinate
		\begin{gather*}
			\psi(t)=f(0,t)=0\,\,\forall\,\, t \in \R
	 		\\
			\partial_yf(0,0)=\psi'(0)=0
		\end{gather*}
	\end{itemize}
	
	$\Rightarrow f$ è derivabile in $(0,0)$ e $\nabla f(0,0)=(0,0)$. 
	
	Ma $\lim_{(x,y)\rightarrow(0,0)}f(x,y)$ non esiste perché non esiste $\lim_{(x,y)\rightarrow(0,0)}\sin\frac{1}{xy}$ $\Rightarrow f$ non è continua in $(0,0)$, anche se è derivabile.  
\end{example}
\end{exbar}


\begin{exbar}
	{$f:\R^2\rightarrow\R$
		\begin{equation*}
			f(x,y)=
			\begin{cases}
				1&\text{  se  }x^4<y<x^2
				\\
				0 &\text{  altrimenti}
			\end{cases}
		\end{equation*}
		
		\image[0.7]{calcolo_differenziale/pag335.png} % pag 335
		$${\color{Goldenrod} \lim_{t\to0} \frac{f\left( (0,0) + t\bar{v} \right) - f(0,0)}{t} = *}$$
		
		$\Rightarrow D_{\overline{v}}f(0,0)=0\,\,\forall$ versore $\overline{v}$ però $f$ non è continua in $(0,0)$.}
\end{exbar}


\subsection{Differenziabilità}

$f:\R\rightarrow\R$ è derivabile in $x_0\in \R \Leftrightarrow \exists \,\, a \in \R \mid f(x)=f(x_0)+a(x-x_0)+o(x-x_0)$ per $x \rightarrow x_0$ e in tal caso $a=f'(x_0)$.

$f:\R^n\rightarrow\R, \overline{x_0}\in\R^n$, $\exists\overline{a}\in\R^n$

$${\color{blue} (*)}f(\overline{x})=\overline{x_0}+ \langle \overline{a},\overline{x}-\overline{x_0} \rangle +o(\overline{x}-\overline{x_0})$$

$$\lim_{\overline{x}\rightarrow\overline{x_0}}\frac{o(\overline{x}-\overline{x_0})}{\|\overline{x}-\overline{x_0}\|}=0$$

Se vale {\color{blue}$(*)$}, $f$ è continua e posso definire l'iperpiano tangente al grafico di $f$ in $(\overline{x_0},f(\overline{x_0}))$ come l'iperpiano $\R^{n+1}$ di equazione
\begin{equation*}
	x_{n+1}=f(\overline{x_0})+\langle \overline{a},\overline{x}-\overline{x_0}\rangle
\end{equation*}
\begin{equation*}
	f(\overline{x})-f(\overline{x_0})-\langle \overline{a},\overline{x}-\overline{x_0} \rangle=o(\overline{x}-\overline{x_0})
\end{equation*}

\textcolor{blue}{$\Gamma_f=\{(\overline{x},f(\overline{x}))\mid \overline{x} \in \R^n\}\subseteq\R^{n+1}$, $(x_1,...,x_n,x_{n+1})$}


\begin{definition}
	$A\subseteq\R^n$ aperto, $f:A\rightarrow \R$, $\overline{x_0}\in A$. $f$ si dice \textbf{differenziabile} in $\overline{x_0}$ se $\exists\,\, \overline{a}\in \R^n$ tale che
	\begin{equation*}
		\lim_{\overline{x}\rightarrow\overline{x_0}}\frac{f(\overline{x})-f(\overline{x_0})-\langle \overline{a},\overline{x}-\overline{x_0}\rangle}{\|\overline{x}-\overline{x_0}\|}=0.
	\end{equation*}
\end{definition}


\begin{theorem}
	\label{th: pag 337}
	
	$A\subseteq\R^n$ aperto, $f:A\rightarrow\R$ differenziabile in $\overline{x_0}\in A$. Allora 
	\begin{enumerate}
		\item $f$ è continua e derivabile in $\overline{x_0}$;
		\item vale 
		\begin{equation*}
			\lim_{\overline{x}\rightarrow\overline{x_0}}\frac{f(\overline{x})-f(\overline{x_0})-\langle \nabla f(\overline{x_0}),\overline{x}-\overline{x_0}\rangle}{\|\overline{x}-\overline{x_0}\|}=0;
		\end{equation*}
		\item (Formula del gradiente) $\forall$ versore $\overline{v}\in\R^n\exists$ la derivata direzionale di $f$ lungo $\overline{v}$ in $\overline{x_0}$ e vale
		\begin{equation*}
			D_{\overline{v}}f(\overline{x_0})=\langle \nabla f(\overline{x_0}), \overline{v} \rangle.
		\end{equation*}
	\end{enumerate}
\end{theorem}


\begin{attbar}
	Se $f$ è funzione di una variabile reale, $f: \dom f\rightarrow\R$, $\dom f \subseteq\R$, $f$ è differenziabile in $x_0\in \dom f \Leftrightarrow$ è derivabile in $x_0$.
\end{attbar}


\textbf{Osservazione:} la tesi 2) del teorema dice che il vettore $\overline{a}$ he compare nella definizione di differenziabilità è $\nabla f(\overline{x_0})$. Quindi, si può riparafrasare la definizione di differenziabilità dicendo che $f$ è differenziabile in $\overline{x_0}$ se è derivabile in $\overline{x_0}$ e vale 
\begin{equation*}
	\lim_{\overline{x}\rightarrow\overline{x_0}}\frac{f(\overline{x})-f(\overline{x_0})-\langle \nabla f(\overline{x_0}),\overline{x}-\overline{x_0}\rangle}{\|\overline{x}-\overline{x_0}\|}=0.
\end{equation*}


\begin{dembar}
	\textbf{Dimostrazione} del \textbf{Teorema \ref{th: pag 337}}
	
	\begin{enumerate}
		\item $f(\overline{x})=f(\overline{x_0})+\langle \overline{a},\overline{x}-\overline{x_0} \rangle+o(\overline{x}-\overline{x_0})$ $\exists \ \overline{a}\in \R^n$
		
		$$\lim_{\overline{x}\rightarrow\overline{x_0}}f(\overline{x})=\lim_{\overline{x}\rightarrow\overline{x_0}}f(\overline{x_0})+ \lowercomment{\langle \overline{a},\overline{x}-\overline{x_0} \rangle}{\myarrow[270]}{0} +\lowercomment{o(\overline{x}-\overline{x_0})=f(\overline{x_0})}{\myarrow[270]}{0}$$
		
		$\Rightarrow f$ è continua in $\overline{x_0}$.
		
		Calcoliamo $\partial_{x_k}f(\overline{x_0})$
		\begin{gather*} 
			\partial_{x_k}f(\overline{x_0})=\lim_{\overline{x}\rightarrow\overline{x_0}}\frac{f(\overline{x_0}+t\overline{e}^k)-f(\overline{x_0})}{t}
			\\
			f(\overline{x})=f(\overline{x_0})+\langle\overline{a},\overline{x}-\overline{x_0}\rangle+o(\overline{x}-\overline{x_0})
			\\
			{\color{blue} \myarrow[270] \overline{x} = \overline{x}_0 + \overline{e}_k}
			\\
			f(\overline{x_0}+t\overline{e_k})=f(\overline{x_0})+\langle\overline{a},t\overline{e_k}\rangle+o(t\overline{e_k})=f(\overline{x_o})+ta_k+o(t)
		\end{gather*}
		
		dove $\overline{a}=(a_1,...,a_n)$, cioè $a_k$ è la $k$-esima componente del vettore $\overline{a}$.
		
		$$\partial_{x_k}f(\overline{x_0})=\lim_{\overline{x}\rightarrow\overline{x_0}}\frac{f(\overline{x_0})-ta_k+o(t)-f(\overline{x_0})}{t}=a_k$$
		
		$\Rightarrow f$ è derivabile in $\overline{x_0}$ e $\nabla f(x_0)=\overline{a}$.
		
		\item \'E conseguenza del fatto che 
		$$\lim_{\overline{x}\rightarrow\overline{x_0}}\frac{f(\overline{x})-f(\overline{x_0})-\langle\overline{a},\overline{x}-\overline{x_0}\rangle}{\|\overline{x}-\overline{x_0}\|}=0$$ 
		
		e che $\overline{a}=\nabla f(\overline{x_0})$.
		
		In particolare, 
		
		$${\color{blue}(*)} f(\overline{x})=f(\overline{x_0})+\langle \nabla f(\overline{x_0}),\overline{x}-\overline{x_0} \rangle +o(\overline{x}-\overline{x_0})$$
		
		\item Sia $\overline{v}\in \R^n$ versore. Calcoliamo $D_{\overline{v}}f(\overline{x_0})$
		
		$$D_{\overline{v}}f(\overline{x_0})=\lim_{t\rightarrow0}\frac{f(\overline{x_0}+t\overline{v})-f(\overline{x_0})}{t}$$
		
		Utilizziamo {\color{blue}$(*)$} con $\overline{x}=\overline{x_0}+t\overline{v}$
		
		$$f(\overline{x_0}+t\overline{v})=f(\overline{x_0})+\langle \nabla f(\overline{x_0}), t\overline{v} \rangle + o(t\overline{v})=f(\overline{x_0})+t\langle \nabla f(\overline{x_0}),\overline{v} \rangle +o(t)$$
		
		$$D_{\overline{v}}f(\overline{x_0})=\lim_{t\rightarrow 0}\frac{ \cancel{f(\overline{x_0})}+t\langle\nabla f(\overline{x_0}),\overline{v}\rangle+o(t)-\cancel{f(\overline{x_0})}}{t}=\langle \nabla f(\overline{x_0}),\overline{v} \rangle. \qquad \square$$
	\end{enumerate}
\end{dembar}


\begin{attbar}
	Una funzione può essere continua e derivabile in un punto senza essere differenziabile.
\end{attbar}


\begin{exbar}
	\begin{equation*}
		f(x,y)=\begin{cases}
			\frac{x^2y}{x^2+y^2}&\text{  se  }(x,y)\neq (0,0) \\
			0&\text{  se  }(x,y)=(0,0)
		\end{cases}
	\end{equation*}
	
	$$0 \leq |f(x,y)|=\frac{x^2|y|}{x^2+y^2}\leq |y|\xrightarrow{(x,y)\rightarrow(0,0)}0$$
	
	$\Rightarrow f$ è continua in $(0,0)$. 
	
	$f(x,0)=f(0,y)=0 \,\,\forall \,\,x,y \in \R$, $\partial_{x}f(0,0)=\partial_yf(0,0)=0$
	
	$\Rightarrow f$ è derivabile in $(0,0)$ e $\nabla f(0,0)=(0,0)$.
	
	$f$ è differenziabile in $(0,0)$?
	
	Lo è $\Leftrightarrow$
	
	$$\lim_{(x,y)\rightarrow(0,0)}\frac{f(x,y)-\uppercomment{f(0,0)}{}{=0}-\uppercomment{\langle \nabla f(0,0),(x,y)-(0,0) \rangle}{}{=0}}{\|(x,y)-(0,0)\|}=0$$
	
	Dobbiamo calcolare 
	
	$$\lim_{(x,y)\rightarrow(0,0)}\frac{f(x,y)}{\sqrt{x^2+y^2}}=\lim_{(x,y)\rightarrow(0,0)}\uppercomment{\frac{x^2y}{(x^2+y^2)\sqrt{x^2+y^2}}}{}{\phi(x,y)}$$
	
	Passiamo in coordinate polari
	
	$$\phi(\rho\cos\theta,\rho\sin\theta)=\frac{\rho^3\cos^2\theta\sin\theta}{\rho^2\cdot \rho}=\cos^2\theta\sin\theta$$
	
	$\lim_{\rho\rightarrow0}\phi(\rho\cos\theta,\rho\sin\theta)=\cos^2\theta\sin\theta$, che dipende da $\theta$
	
	$\Rightarrow \lim_{(x,y)\rightarrow(0,0)}\frac{f(x,y)}{\sqrt{x^2+y^2}}$ non esiste $\Rightarrow f$ non è differenziabile in $(0,0)$.
\end{exbar}


\begin{attbar}
	Somma, prodotto, quoziente (con denominatore non nullo) e composizioni di funzioni differenziabili sono differenziabili.
\end{attbar}


\textbf{Osservazione (direzione di massima crescita):}

$f:\R^2\rightarrow\R$, differenziabile in $(x_0,y_0)$, consideriamo tutte le rette passanti per $(x_0,y_0)$ e le restrizioni di $f$ lungo tali rette.

Qual è la retta lungo cui $f$ cresce più velocemente?

\image{calcolo_differenziale/pag342.png} % pag 342

$f:A \rightarrow\R$, $A\subseteq\R^n$ aperto, $\overline{x_0}\in A$, $f$ differenziabile in $\overline{x_0}$. Voglio trovare il versore $\overline{v}$ di $\R^n$ per cui la restrizione di $f$ lungo la retta per $\overline{x_0}$ parallela a $\overline{v}$

$\phi_{\overline{v}}t\mapsto f(\overline{x_0}+\overline{v})$ cresce più velocemente in un intorno di $\overline{x_0}$, $\phi_{\overline{v}}'(0)=D_{\overline{v}}f(\overline{x_0})$.

Cerco $\overline{v}$ in modo che $\phi_{\overline{v}}'(0)=D_{\overline{v}}f(\overline{x_0})$ sia massima

$$D_{\overline{v}}f(\overline{x_0})=\langle \nabla f(x_0),\overline{v} \rangle$$ è massima se $\overline{v}=\frac{\nabla f(\overline{x_0})}{\|\nabla f(\overline{x_0})\|}$ ($\nabla f(\overline{x_0})\neq \overline{0}$).

Quindi $\nabla f(\overline{x_0})$ fornisce la direzione di massima crescita di $f$ in $\overline{x_0}$.

\image{calcolo_differenziale/pag343.png} % pag 343


\begin{definition}
	$A\subseteq\R^n$ aperto, $f:A \rightarrow \R$ differenziabile in $\overline{x_0}\in A$. Si definisce \textbf{iperpiano tangente} al grafico di $f$ in $(\overline{x_0},f(\overline{x_0}))$ l'iperpiano di $\R^{n+1}$ di equazione
	\begin{gather*}
		x_{n+1}=f(x_0)+\langle \nabla f(\overline{x_0}), \overline{x}-\overline{x_0} \rangle
		\\
		\overline{x}(x_1,...,x_n).
	\end{gather*}
	
	In particolare, se $A \subseteq \R^2$ e $f$ è differenziabile in $(x_0,y_0)$, il piano tangente al grafico di $f$ in $((x_0,y_0),f(x_0,y_0))$ è il piano di equazione
	\begin{align*}
		z&=f(x_0,y_0)-\langle \nabla f(x_0,y_0),(x-x_0,y-y_0) \rangle
		\\
		\text{cioè }z&=f(x_0,y_0)+\partial_xf(x_0,y_0)(x-x_0)+\partial_yf(x_0,y_0)(y-y_0).
	\end{align*}
\end{definition}


\begin{exbar}
\begin{example}
	$$f(x,y)=e^{x+y-1}+x^4+y^2$$
	
	Calcoliamo l'equazione del piano tangente al grafico di $f$ in $((1,0),f(1,0))$.
	
	$$z=f(1,0)+\partial_xf(1,0)(x-1)+\partial_yf(1,0)y$$
	
	$f(1,0)=1+1=2$
	
	$$\partial_xf(x,y)=e^{x+y-1}+4x^3$$
	
	$\partial_x f(1,0)=1+4=5$
	
	$$\partial_yf(x,y)=e^{x+y-1}+2y$$
	
	$\partial_yf(1,0)=1$
	
	$$z=2+5(x-1)+y \qquad \Rightarrow \qquad z=5x+y-3$$
\end{example}
\end{exbar}


\begin{theorem} \textbf{Teorema del differenziale totale}
	
	$A \subseteq \R^n$, $f:A\rightarrow \R$ derivabile in un intorno di $\overline{x_0}\in A$. Se le derivate parziali di $f$ sono continue in $\overline{x_0}$, cioè
	\begin{equation*}
		\lim_{\overline{x}\rightarrow\overline{x_0}}\partial_{x_k}f(\overline{x})=\partial_{x_k}f(\overline{x_0})\,\,\, \forall \,\,k=1,...,n,
	\end{equation*}
	
	allora $f$ è differenziabile in $\overline{x_0}$.
\end{theorem}


\begin{exbar}
	$$f(x,y)=e^{x+y-1}+x^4+y^2$$
	
	è derivabile in $\R^2$ e 
	
	$$\partial_x f(x,y)=e^{x+y-1}+4x^3, \qquad \partial_y f(x,y)=e^{x+y-1}+2y^2$$
	
	sono funzioni continue in $\R^2 \Rightarrow f$ è differenziabile in ogni punto di $\R^2$.
\end{exbar}


\begin{definition}
	Una funzione $f$ definita in un sottoinsieme di $\R^n$ si dice differenziabile in $A \subseteq\R^n$ se è differenziabile in ogni punto di $A$. Si dice di classe $C^1$ in $A$ e si scrive $f \in C^1(A)$ se è derivabile in $A$ e le sue derivate parziali sono continue in $A$.
\end{definition}


\begin{corollary}
	$A\subseteq\R^n$ aperto, $f\in C^1(A)$. Allora $f$ è differenziabile in $A$.
\end{corollary}


\begin{attbar}
	Il teorema del differenziale totale fornisce una condizione sufficiente, \textbf{ma non necessaria}, per la differenziabilità, cioè una funzione può essere differenziabile in un punto senza avere le derivate parziali continue in quel punto.
\end{attbar}


\begin{exbar}
	\begin{equation*}
		f(x,y)=
		\begin{cases}
			x^2\sin\frac{1}{x}&\text{  se  }x\neq0
			\\
			0&\text{  se  }x=0
		\end{cases}
		\qquad (x,y)\in \R^2
	\end{equation*}
	
	$$f\in C^1\left( \R^2\backslash\{(0,y),y\in\R\} \right)$$
	
	Sia $(0,y_0)$ fissato e calcoliamo $\partial_x f(0,y_0)$ e $\partial_y f(0,y_0)$
	
	$\phi: f \mapsto f(t,y_0)=t^2\sin\frac{1}{t}$, $t \neq 0$
	
	$\partial_x f(0,y_0)=\phi'(0)=0$
	
	$\Phi:f\mapsto f(0,y_0+t)=0\,\,\, \forall t$
	
	$\partial_y f(0,y_0)=0=\Phi'(0)$
	
	$x \neq 0$, $\partial_x f(x,y)=2x\sin\frac{1}{x}-\cos\frac{1}{x}$
	
	$\lim_{(x,y)\rightarrow(0,y_0)}\partial_x f(x,y)$ non esiste $\Rightarrow \partial_xf$ non è continua in $(0,y_0)$. 
	
	Ciononostante è differenziabile in $(0,y_0)$
	
	$\nabla f(0,y_0)=(0,0)$.
	
	Devo verificare che 
	
	$$\lim_{(x,y)\rightarrow(0,y_0)}\frac{f(x,y)-\uppercomment{f(0,y_0)}{}{=0}-\uppercomment{\langle \nabla f(0,y_0),(x,y-y_0) \rangle}{}{=0}}{\|(x,y)-(0,y_0)\|}=0$$
	
	$$\lim_{(x,y)\rightarrow(0,y_0)}\frac{x^2\sin\frac{1}{x}}{\sqrt{x^2+(y-y_0)^2}}$$
	
	$$ \bigg| \frac{x^2\sin\frac{1}{x}}{\underbrace{\sqrt{x^2+(y-y_0)^2}}_{{\color{blue} \geq x^2}}} \bigg|\leq \frac{x^2}{|x|} \bigg|\sin\frac{1}{x} \bigg|=|x| \bigg|\sin\frac{1}{x} \bigg|\xrightarrow{(x,y)\rightarrow(0,y_0)}0$$.
\end{exbar}


\begin{exbar}
\begin{example}
	Data la funzione 
	
	$$f(x,y)=\ln(2+\sqrt{2x^2+y^2})\sin(x+2y)$$ 
	
	se ne determinino dominio, punti di continuità, di derivabilità e di differenziabilità.
	\begin{enumerate}
		\item Dominio
		
		$2+\sqrt{2x^2+y^2}>0$ vero $\forall(x,y)\in\R^2 \Rightarrow \dom f =\R^2$
		
		\item Derivabilità
		
		Per $2x^2+y^2\neq 0$, cioè in $\R^2\backslash\{(0,0)\}$, $f$ è prodotto e composizione di funzioni derivabili, e quindi è derivabile.
		\begin{align*} 
			\partial_x f(x,y)&=\frac{\sin(x+2y)}{2+\sqrt{2x^2+y^2}}\cdot \frac{1}{2\sqrt{2x^2+y^2}}\cdot4x+\ln(2+\sqrt{2x^2+y^2})\cos(x+2y)
			\\
			\partial_yf(x,y)&=\frac{\sin(x+2y)}{2+\sqrt{2x^2+y^2}}\cdot\frac{1}{2\sqrt{2x^2+y^2}}\cdot 2y+2\ln(2+\sqrt{2x^2+y^2})\cos(x+2y)
		\end{align*}
		
		$\Rightarrow \partial_x f$ e $\partial_y f$ sono continue in $\R^2\backslash\{(0,0)\} \Rightarrow f \in C^1(\R^2\backslash\{(0,0)\})$ e quindi è differenziabile in $\R^2\backslash\{(0,0)\}$.
		
		Per verificare se $f$ è derivabile in $(0,0)$ utilizzo la definizione di derivata parziale, guardando le restrizioni di $f$ lungo gli assi cartesiani.
		
		$\phi: t \mapsto f(t,0)=\ln(2+\sqrt{2t^2})\sin(t)= \ln(2+\sqrt{2}|t|)\sin(t)$.
		
		Se esiste, $\partial_x f(0,0)=\phi'(0)$
		
		Se $t\neq 0$
		
		$$\phi'(t)=\frac{\sin(t)}{2+\sqrt{2}|t|}\sqrt{2}\sgn(t)+\ln(2+\sqrt{2}|t|)\cos(t)$$
		
		$\lim_{t \rightarrow0} \phi'(t)=\ln 2 \Rightarrow \phi$ è derivabile in $0$ e $\phi'(0)=\ln 2=\partial_x f(0,0)$
		
		$\psi: t \mapsto f(0,t)=\ln(2+\sqrt{t^2})\sin(2t)=\ln(2+|t|)\sin(2t)$.
		
		Se esiste, $\partial_y f(0,0)=\psi'(0)$
		
		Se $t \neq 0$
		
		$$\psi'(t)=\frac{\sin(2t)}{2+|t|}\sgn (t)+2\ln(2+|t|)\cos(2t)$$
		
		$\lim_{t\rightarrow0}\psi'(t)=2\ln 2=\Phi'(0)=\partial_y f(0,0)$
		
		$\Rightarrow f$ è derivabile in $(0,0)$ e $\nabla f(0,0) =(\ln2,2\ln2)$
		
		\item Differenziabilità in $(0,0)$
		
		Conviene utilizzare il teorema del differenziale totale e provare a verificare che 
		{\centering $\lim_{(x,y)\rightarrow(0,0)}\partial_xf(x,y)=\partial_xf(0,0)=\ln2$ e $\lim_{(x,y)\rightarrow(0,0)}\partial_y f(x,y)=\partial_y f(0,0)=2 \ln 2$ \par}
		
		Oppure conviene utilizzare la definizione di differenziabilità e calcolare
		
		$$\lim_{(x,y)\rightarrow(0,0)}\frac{f(x,y)-f(0,0)-\langle\nabla f(0,0),(x,y)\rangle}{\sqrt{x^2+y^2}}?$$
		
		{\color{red} Conviene utilizzare la definizione.}
		
		Calcoliamo 
		\begin{align*} 
			&\lim_{(x,y)\rightarrow(0,0)}\frac{\ln(2+\sqrt{2x^2+y^2})\sin(x+2y)-\ln(2x)-2\ln(2y)}{\sqrt{x^2+y^2}}
			\\
			=&\lim_{(x,y)\rightarrow(0,0)}\frac{\ln(2+\sqrt{2x^2+y^2})\sin(x+2y)-\ln2(x+2y)}{\sqrt{x^2+y^2}}=
			\\
			=&\lim_{(x,y)\rightarrow(0,0)}\frac{1}{\sqrt{x^2+y^2}} \big[ \ln(2+\sqrt{2x^2+y^2})\sin(x+2y)-\ln2\sin(x+2y)+
			\\
			& \qquad \qquad +\ln2\sin(x+2y)-\ln2(x+2y) \big]=
			\\
			=&\lim_{(x,y)\rightarrow(0,0)} \left[ \frac{\ln(2+\sqrt{2x^2+y^2})-\ln 2}{\sqrt{x^2+y^2}}\sin(x+2y) + \ln 2 \frac{\sin(x+2y)-(x+2y)}{\sqrt{x^2+y^2}}\right]
		\end{align*}
		
		Dimostriamo che 
		
		$$\lim_{(x,y)\rightarrow(0,0)} \uppercomment{\frac{\ln(2+\sqrt{2x^2+y^2}-\ln2)}{\sqrt{x^2+y^2}}\sin(x+2y)}{}{\psi(x,y)}=0$$
		\begin{align*} 
			\ln(2+\sqrt{2x^2+y^2})-\ln2 
			&=\ln\left( \frac{2+\sqrt{2x^2+y^2}}{2}\right)=\ln\left(1+\frac{\sqrt{2x^2+y^2}}{2}\right)=
			\\
			&=\frac{1}{2}\sqrt{2x^2+y^2}+o(\sqrt{2x^2+y^2})
		\end{align*}
			
		$$\lim_{(x,y)\rightarrow(0,0)} \psi(x,y) \PdS \lim_{(x,y)\rightarrow(0,0)}\frac{1}{2}\frac{\sqrt{2x^2+y^2}}{\sqrt{x^2+y^2}} \uppercomment{\sin(x+2y)}{}{\to 0}=0$$
		
		{\color{blue} $\left| \frac{\sqrt{2x^2+y^2}}{x^2+y^2} \right| = \sqrt{\frac{2x^2+y^2}{x^2+y^2}} \leq \sqrt{\frac{2x^2+2y^2}{x^2+y^2}} = \sqrt{2}$
		}
		
		Dimostriamo anche
		
		$$\lim_{(x,y)\rightarrow(0,0)}\frac{\sin(x+2y)-(x+2y)}{\sqrt{x^2+y^2}}=0$$
		
		$\sin(x+2y)=(x+2y)-\frac{1}{6}(x+2y)^3+o((x+2y)^4)$
		
		$$\lim_{(x,y)\rightarrow(0,0)}\frac{\sin(x+2y)-(x+2y)}{\sqrt{x^2+y^2}} \PdS \lim_{(x,y)\rightarrow(0,0)}-\frac{1}{6} \uppercomment{\frac{(x+2y)^3}{\sqrt{x^2+y^2}}}{}{\xi(x,y)}$$
		
		Passando in coordinate polari
		
		$$\xi(\rho\cos\theta,\rho\sin\theta)=\frac{\rho^3(\cos\theta+2\sin\theta)^3}{\rho}=\rho^2(\cos\theta+2\sin\theta)^3\xrightarrow{\rho\rightarrow 0}0 \forall \theta$$
		
		$|\xi(\rho\cos\theta,\rho\sin\theta)|\leq 2\rho^2$
		
		$$\sup_{\theta \in [0,2\pi[}|\xi(\rho\cos\theta,\rho\sin\theta)|\leq 27\rho^2\xrightarrow{\rho \rightarrow 0}0$$
		
		$\Rightarrow\lim_{(x,y)\rightarrow(0,0)}-\frac{1}{6}\xi(x,y)=0$  e $f$ è differenziabile in $(0,0)$.
	\end{enumerate}
\end{example}
\end{exbar}

\newpage %riq mal
\begin{attbar}
	\textbf{Notazione:} $\overline{x_1},\overline{x_2}\in\R^n$
	\begin{align*} 
		[\overline{x_1},\overline{x_2}] &=\{\overline{x_1}+\lambda(\overline{x_2}-\overline{x_1})\mid\lambda\in[0,1]\}
		\\
		]\overline{x_1},\overline{x_2}[ &=\{\overline{x_1}+\lambda(\overline{x_2}-\overline{x_1})\mid\lambda\in]0,1[\}
		\\
		[\overline{x_1},\overline{x_2}[ &=\{\overline{x_1}+\lambda(\overline{x_2}-\overline{x_1})\mid\lambda\in[0,1[\}
		\\
		]\overline{x_1},\overline{x_2}] &=\{\overline{x_1}+\lambda(\overline{x_2}-\overline{x_1})\mid\lambda\in]0,1]\}
	\end{align*}
	
	segmenti di estremi $\overline{x_1}$ e $\overline{x_2}$.
\end{attbar}


\begin{theorem} \textbf{Teorema del valore medio}
	
	\label{th: pag 355}
	$A \subseteq \R^n$ connesso per archi, $f:A\rightarrow \R$, $\overline{x_1},\overline{x_2}\in A$ tali che $[\overline{x_1},\overline{x_2}]\subseteq A$. Se $f$ è continua in $[\overline{x_1},\overline{x_2}]$ e differenziabile in $]\overline{x_1},\overline{x_2}[$, allora $\exists\,\, \overline{\xi} \in]\overline{x_1},\overline{x_2}[$ tale che $f(\overline{x_2})-f(\overline{x_1})=\langle \nabla f(\overline{\xi}),\overline{x_2}-\overline{x_1} \rangle$.
\end{theorem}


\begin{dembar}
	\textbf{Dimostrazione} del \textbf{Teorema \ref{th: pag 355}}
		
	Sia $\overline{v}=\frac{\overline{x_2}-\overline{x_1}}{\|\overline{x_2}-\overline{x_1}\|}$ versore.
	
	$$\phi_{\overline{v}}:[0,\|\overline{x_2}-\overline{x_1}\|]\rightarrow\R \qquad \phi_{\overline{v}}(t)=f(\overline{x_1}+t\overline{v})$$
	
	$\phi_{\overline{v}}$ è continua nel suo dominio e derivabile in $]0,\| \overline{x_2}-\overline{x_1} \|[$ per la formula del gradiente e $\phi_{\overline{v}}'(t)=\langle\nabla f(\overline{x_1}+t\overline{v}),\overline{v}\rangle$.
	
	Per il teorema di Lagrange $\exists\,\, \overline{t}\in]0,\|\overline{x_2}-\overline{x_1}\|[\,\, \mid $
	\begin{gather*} 
		\lowercomment{\phi_{\overline{v}}(\|\overline{x_2}-\overline{x_1}\|)}{=f(\overline{x}_2)}{}- \lowercomment{\phi(0)}{}{=f(\overline{x}_1) }=\phi'(\overline{t})(\|\overline{x_2}-\overline{x_1}\|-0)
		\\
		{
			\color{blue} \phi' (\overline{t} = \langle \nabla f (\underbrace{\overline{x}_1 + \overline{t} \overline{v}}_{{\color{teal} = \overline{\xi}}}), \frac{\overline{x}_2 - \overline{x}_1} {\| \overline{x}_2 - \overline{x}_1 \| } \rangle
		}
		\\
		\Rightarrow f(\overline{x}_2) - f(\overline{x}_1) = \langle \nabla f(\overline{\xi}), \overline{x}_2 - \overline{x}_1 \rangle \qquad \square
	\end{gather*}
\end{dembar}


\begin{corollary}
	\label{cor: pag 357}
	Sia $A \subseteq \R^n$ aperto connesso per archi, $f:A \rightarrow \R$ derivabile con $\nabla f(\overline{x})=\overline{0}$ $\forall \,\, \overline{x} \in A$. Allora $f$ è costante. 
\end{corollary}


\begin{dembar}
	\textbf{Dimostrazione} del \textbf{Corollario \ref{cor: pag 357}}
	
	$f \in C^1(A)$ e quindi è differenziabile in $A$. Supponendo per semplicità $A$ convesso. $\overline{x_1},\overline{x_2}\in A\Rightarrow [\overline{x_1},\overline{x_2}]\subseteq A \,\,\exists \,\, \overline{\xi}\in ]\overline{x_1},\overline{x_2}[\,\,\mid\,\, f(\overline{x_2})-f(\overline{x_1})=\langle\nabla f(\overline{\xi}),\overline{x_2}-\overline{x_1}\rangle =0 \Rightarrow f$ è costante.
	
	\image{calcolo_differenziale/pag357.png} % pag 357
	
	$f(\overline{x_1})=f(\overline{y_1})=f(\overline{y_2})=f(\overline{x_2})$.
\end{dembar}


\begin{exbar}
\begin{example}
	Sia data la funzione $f: \R^2\rightarrow \R$
	\begin{equation*}
		f(x,y)=
		\begin{cases}
			\frac{\sin(x|y|^\alpha)}{x^4+|y|^7}& \text{  se  }(x,y)\neq (0,0)
			\\
			0& \text{  se  } (x,y)=(0,0) 
		\end{cases}
	\end{equation*}
	
	Si stabilisca per quali valori del parametro $\alpha >0$
	\begin{enumerate}
		\item $f$ è continua in $(0,0)$
		\item $f$ è derivabile in $(0,0)$
		\item $f$ è differenziabile in $(0,0)$
	\end{enumerate}
	
	{\centering $\sim \circ \sim$ \par}
	
	\begin{enumerate}
		\item $\sin(x|y|^\alpha)=x|y|^\alpha+o(x|y|^\alpha)$ per $(x,y)\rightarrow (0,0)$ perché $x|y|^\alpha \rightarrow 0 $ per $(x,y)\rightarrow (0,0)$
		
		$$\lim_{(x,y)\rightarrow(0,0)}f(x,y) \PdS \lim_{(x,y)\rightarrow(0,0)} \lowercomment{\frac{x|y|^\alpha}{x^4+|y|^7}}{\phi(x,y)}{}$$
		
		Proviamo con ad utilizzare le coordinate polari 
		
		\begin{align*} 
			\phi(\rho\cos\theta,\rho\sin\theta)
			&=\phi^{\alpha+1}\frac{\cos\theta|\sin \theta|^\alpha}{\rho^4(\cos^4\theta+\rho^3|\sin\theta|^7)} =\rho^{\alpha-3}=\frac{\cos\theta|\sin\theta|^\alpha}{\cos^4\theta+\rho^3|\sin\theta|^7} \xrightarrow{\rho \rightarrow 0} 
			\\
			&\xrightarrow{\rho \rightarrow 0}
				\begin{cases}
				0& \text{  se  } \alpha > 3
				\\ 
				\infty& \text{  o qualcosa che dipende da }\theta \text{ se } \alpha \leq 3
				\end{cases}
		\end{align*}
		
		$\Rightarrow f$ non è continua in $(0,0)$ se $\alpha \leq 3$.
		
		Per $\alpha > 3 $ dobbiamo stimare
		
		\begin{align*} 
			\sup_{\theta \in [0,2\pi[} \rho^{\alpha-3}\frac{|\cos\theta||\sin\theta|^\alpha}{\cos^4\theta+\rho^3|\sin\theta|^7}
			&{\color{blue} \leq \rho^{\alpha-3}\frac{|\cos\theta||\sin\theta|^\alpha}{\rho^3|\sin\theta|^7}}=
			\\
			&=\rho^{\alpha-6}|\cos\theta||\sin\theta|^{\alpha-7}\leq \rho^{\alpha-6}
		\end{align*}
		
		{\centering \color{blue}se $\alpha >7$ $\Rightarrow f$ {\color{red}(e non $\Leftrightarrow$)} è continua, quindi non ci porta da nessuna parte. \par}
		
		Proviamo a studiare la restrizione di $\phi$ lungo $x^4 =|y|^7$
		
		$$|\phi(x,y)||_{x^4=|y|^7} = \frac{|y|^{\frac{7}{4}} |y|^\alpha}{2|y|^7}=\frac{1}{2}|y|^{\alpha +\frac{7}{4}-7}=\frac{1}{2}|y|^{\alpha-\frac{21}{4}}$$
		
		Se $\alpha \leq \frac{21}{4} \Rightarrow \phi(x,y) \nrightarrow 0$ per $(x,y)\rightarrow(0,0)$.
		
		Se voglio aver speranza che $\phi(x,y)\rightarrow 0$ per $(x,y)\rightarrow(0,0)$ deve essere $\alpha > \frac{21}{4}$.
		
		Siano $0 \leq p \leq q$, allora $\exists\,\, C >0 \mid$
		
		$$ \underbrace{\frac{|\alpha|^{q-p}|\beta|^p}{|\alpha|^q+|\beta|^q}}_{{\color{blue} \frac{1}{|\alpha|^q+|\beta|^q} \leq \frac{C}{|\alpha|^{q-p} |\beta|^p}}} \leq C \qquad \forall (\alpha,\beta)\neq(0,0)$$
		
		\begin{align*} 
			|\phi(x,y)|
			&=\frac{|x||y|^\alpha}{|x|^4+|y|^7} \lowercomment{=}{q=4}{} \frac{|x||y|^\alpha}{|x|^q+|y|^{\frac{7}{q}\cdot q}}=
			\\
			&= |x||y|^\alpha \cdot \lowercomment{\frac{1}{|x|^q+|y|^{\frac{7}{q}\cdot p}}}{\alpha=|x|}{\beta=|y|^{\frac{7}{q}}} \leq |x||y|^\alpha \frac{C}{|x|^{4-p}\cdot |y|^{\frac{7}{q}p}} = & 0 \leq p \leq 4, 
			\\
			&= C|x|^{p-3}|y|^{\alpha-\frac{7}{4}p} \uppercomment{=}{}{p=3} C|y|^{\alpha-\frac{21}{4}}
		\end{align*}
		{\centering
		$ \exists  C > 0 \mid |\phi(x,y)|\leq C|y|^{\alpha-\frac{21}{4}}\xrightarrow{(x,y)\rightarrow(0,0)} 0$ se $\alpha > \frac{21}{4}$
		
		$\Rightarrow f$ è continua in $(0,0)\Leftrightarrow \alpha> \frac{21}{4}$. \par}
		
		
		{\color{blue} Dimostriamo che $\exists C > 0 \mid $
			$$ \lowercomment{\frac{|\alpha|^{q-p}|\beta|^{p}}{|\alpha|^q+|\beta|^q}}{{\color{teal} \psi(x,y)}}{} \leq C \qquad \forall (\alpha,\beta)\neq(0,0)$$
			
			$\Phi(0,\beta)=0$, $\alpha \neq 0$
			
			\begin{align*} 
				\frac{|\alpha|^{q-p}|\beta|^p}{|\alpha|^q+|\beta|^q}
				&=\frac{|\alpha|^q}{|\alpha^q}\frac{|\alpha|^{-p}|\beta|^p}{1+|\frac{\beta}{\alpha}|^q}=
				\\
				&=\frac{|\frac{\beta}{\alpha}|^p}{1-|\frac{\beta}{\alpha}|^q}\leq C & |\frac{\beta}{\alpha}|\geq 0
			\end{align*}
		}
		
		{\color{orange}{\centering 
			$\xi(t)=\frac{t^p}{1+t^q}$, $t\geq 0$, 
			
			$\exists C > 0 \mid \xi(t)\leq C \,\, \forall\,\, t \geq 0$ perché $0 \leq p \leq q$. \par}		
		}
		
		\item Derivabilità in $(0,0)$.
		
		Guardo le restrizioni di $f$ lungo gli assi cartesiani $f(x,0)=0\,\, \forall x$, $f(0,y)=0\,\, \forall y$
		
		{\centering $\partial_xf(0,0)=0$ e $\partial_yf(0,0)=0$ \par}
		
		perché $f$ è costante lungo gli assi cartesiani $\Rightarrow f$ è derivabile in $(0,0) \forall \,\, \alpha >0$ e $\nabla f(0,0)=(0,0)$.
		
		\item Calcoliamo 
		\begin{align*} 
			\lim_{(x,y)\rightarrow(0,0)}\frac{f(x,y)-\uppercomment{f(0,0)}{}{=0}-\langle \uppercomment{\nabla f(0,0)}{}{=\overline{0}},(x,y) \rangle}{\sqrt{x^2+y^2}}
			&=\lim_{(x,y)\rightarrow(0,0)}\frac{f(x,y)}{\sqrt{x^2+y^2}}=
			\\
			&=\lim_{(x,y)\rightarrow(0,0)} \frac{\uppercomment{\sin(x|y|^\alpha)}{}{= x|y|^\alpha + o(x|y|^\alpha)}}{(x^4+|y|^7)\sqrt{x^2+y^2}} 
			\\
			&\PdS \lim_{(x,y)\rightarrow(0,0)}\frac{x|y|^\alpha}{(x^4+|y|^7)\sqrt{x^2+y^2}}
		\end{align*}
		
		$$|\frac{x|y|^\alpha}{x^4+|y|^7}|\leq C|y|^{\alpha-\frac{21}{4}}$$
		
		$$\bigg|\frac{x|y|^\alpha}{(x^4+y^7)\sqrt{x^2+y^2}}\bigg| \leq \frac{C|y|^{\alpha-\frac{21}{4}}}{\underbrace{\sqrt{x^2+y^2}}_{{\color{blue} \geq y^2}}}\leq C \frac{|y|^{\alpha-\frac{21}{4}}}{|y|}=C |y|^{\alpha -\frac{25}{4}}\xrightarrow{(x,y)\rightarrow(0,0)}0 $$
		
		Se $\alpha > \frac{25}{4}$, $f$ è differenziabile in $(0,0)$.
		
		\begin{align*} 
			\bigg| \frac{x|y|^\alpha}{(x^4+|y|^7)\sqrt{x^2+y^2}} \bigg| \mid_{x^4=|y|^7}
			&=|y|^{\alpha -\frac{21}{4}}\frac{1}{\sqrt{x^2+y^2}}|_{x^4=|y|^7}=|y|^{\alpha-\frac{21}{4}}\frac{1}{\sqrt{|y|^{\frac{7}{2}}+y^2}}=
			\\
			&=\frac{|y|^{\alpha-\frac{21}{4}}}{|y|}\frac{1}{\sqrt{1+|y|^{\frac{3}{2}}}}=|y|^{\alpha-\frac{25}{4}}\frac{1}{\sqrt{1+|y|^{\frac{3}{2}}}}\nrightarrow 0
			\\
			&\text{ per } y \rightarrow 0 \text{ se } \alpha \leq \frac{25}{4}
		\end{align*}
		
		$\Rightarrow f$ non è differenziabile in $(0,0)$ se $\alpha \leq \frac{25}{4}$. $f$ è differenziabile in $(0,0)\Leftrightarrow \alpha > \frac{25}{4}$. 
	\end{enumerate}
\end{example}
\end{exbar}


\subsection{Ricerca di punti di estremo}

$f:]a,b[\rightarrow \R$, derivabile.
\begin{enumerate}
	\item Teorema di Fermat: i punti di massimo e minimo di $f$ sono punti stazionari, cioè soddisfano $f'(x)=0$.
	$f: A \rightarrow \R$, $A \subseteq \R^n$ aperto $\overline{x_0}\in A$ punto di estremo per $f \Rightarrow \nabla f(\overline{x_0})=\overline{0}$
	
	\item Se $x_0 \in ]a,b[$ è punto stazionario, come faccio a capire se è di massimo, di minimo o di flesso orizzontale?
	\begin{itemize} 
		\item $f''(x_0)<0 \Rightarrow x_0$ punto di massimo
		\item $f''(x_0)>0 \Rightarrow x_0$ punto di minimo
		\item $f''(x_0)=0 \Rightarrow x_0$ è candidato punto di flesso.
	\end{itemize}
\end{enumerate}


\begin{definition}
	$f:\dom f \rightarrow \R$, $\dom f \subseteq \R^n$. $\overline{x_0}\in \dom f$ si dice punto di massimo relativo per $f$ se $\exists r >0 \mid$
	
	$$ f(\overline{x_0})\geq f(\overline{x})\forall\overline{x}\in B_r(\overline{x_0})\cap \dom f$$
	
	$\overline{x_0}\in \dom f$ si dice punto di minimo relativo se $\exists r >0 \mid$
	 
	$$ f(\overline{x_0})\leq f(\overline{x})\forall \overline{x} \in B_r(\overline{x_0})\cap \dom f$$
	 
	$\overline{x_0} \in \dom f$ di massimo o minimo relativo si dice anche punto di estremo relativo.
	
	$\overline{x_0}\in \dom f$ si dice di massimo (assoluto o globale) se 
	
	$$f(\overline{x_0})\geq f(\overline{x})\forall \overline{x} \in \dom f$$
	
	Definizione analoga si dà per un punto di minimo (assoluto o globale).
\end{definition}


\begin{theorem} \textbf{di Fermat}
	
	\label{th: pag 366}
	$A \in \R^n$ aperto, $f: A \rightarrow \R$ derivabile in $\overline{x_0} \in A$, punto di estremo relativo per $f$. Allora $\nabla f(\overline{x_0})=\overline{0}$, cioè $\partial_{x_k}f(\overline{x_0})=0 \forall k =1,...,n$.
\end{theorem}


\begin{dembar}
	\textbf{Dimostrazione} del \textbf{Teorema \ref{th: pag 366}}
	
	$\exists \delta >0 \mid B_\delta (\overline{x_0})\subseteq A$ perché $A$ è aperto. Fissato $k=1,...,n$ sia 
	
	$$\phi(t)=f(\overline{x_0}+t\overline{e_k}), \qquad t \in ]-\delta,\delta[$$
	
	restrizione di $f$ lungo un segmento parallelo  a $\overline{e_k}$ passante per $\overline{x_0}$ e di lunghezza $2\delta$.

	\image{calcolo_differenziale/pag366.png} % pag 366

	Per fissare le idee assumiamo che $\overline{x_0}$ sia di massimo relativo. Prendendo $\delta$ abbastanza piccolo, possiamo assumere che 
	
	$$f(\overline{x_0})\geq f(\overline{x})\forall \overline{x} \in B_\delta (\overline{x_0})$$
	
	$$\phi(0)=f(\overline{x_0})\geq f(\overline{x_0}+t\overline{e_k})=\phi(t)\forall \,\, t \in ]-\delta,\delta[$$ 
	
	perché $\overline{x_0}+t \overline{e_k} \in B_\delta (\overline{x_0})\forall t \in ]-\delta,\delta[$ $\Rightarrow t=0$ è punto di massimo per \\%riq mal
	$\phi \Rightarrow \phi'(0)\lowercomment{=}{}{\text{per definizione}}\partial_{x_k}f(\overline{x_0})=0$ per il teorema di Fermat in una variabile. $\qquad\square$
\end{dembar}


\begin{definition}
	$f:\dom f \rightarrow \R$, $\overline{x_0}\in \dom f \subseteq \R^n$. $f$ derivabile in $\overline{x_0}$. Se $\nabla f(\overline{x_0})=\overline{0}$, allora $\overline{x_0}$ si dice \textbf{punto stazionario o critico} per $f$.
\end{definition}


\begin{definition}
	$f: \dom f \rightarrow \R$, $ \overline{x_0}\in \dom f \subseteq \R^n$ punto stazionario. Se $\overline{x_0}$ non è punto di estremo relativo, allora si dice \textbf{punto di sella} per $f$.
\end{definition}


\begin{exbar}
	{\centering $f(x,y)=x^2-y^2$
		
	$\nabla f(x,y)=(2x,-2y)$
	
	$\nabla f(0,0)=(0,0)$
	
	$f(x,0)=x^2\geq 0=f(0,0) \forall x \in \R$
	
	$f(0,y)=-y^2 < 0 =f(0,0)\forall y \neq 0$ \par}
	
	$\Rightarrow (0,0)$ non è punto di estremo relativo per $f$ perché in ogni intorno di $(0,0)$ $f$ assume valori sia maggiori che minori di $0=f(0,0)$.
\end{exbar}


\subsection{Derivate di ordine superiore}

$f: \R \rightarrow \R$, $x_0\in\R \mid f'(x_0)=0$. Guardo $f''(x_0)$.

\image[0.7]{calcolo_differenziale/pag369.png} % pag 369
\begin{align*} 
	f(x)
	&=f(x_0)+\uppercomment{f'(x_0)(x-x_0)}{}{=0}+\frac{1}{2}f''(x_0)(x-x_0)^2+o((x-x_0)^2)
	\\
	&=f(x_0)+\frac{1}{2}f''(x_0)(x-x_0)^2+{\color{blue}\xcancel{o((x-x_0)^2)}}
\end{align*}
$$y=f(x_0)+\frac{1}{2}f''(x_0)(x-x_0)^2$$
\begin{align*} 
	f(x)
	&=f(x_0)+\frac{1}{2}f''(x_0)(x-x_0)^2+o((x-x_0)^2)=
	\\
	&=\lowercomment{ 
		f(x_0)+(x-x_0)^2 \bigg[ \underbrace{\frac{1}{2} f''(x_0)+o(1)}_{{\color{teal} \genfrac{}{}{0pt}{}{>0 \textbf{ o } < 0 \text{ vicino a}}{x_0 \text{ se } f''(x_0) > 0 \text{ o } f''(x_0)<0} }} \bigg]}
		{>f(x_0) \text{ o } < f(x_0) \text{ vicino a }x_0}{\text{se } f''(x_0)<0 \text{ o } f''(x_0)>0}
\end{align*} 

Il criterio della derivata seconda per studiare la natura di un punto critico discende dalla possibilità di scrivere una formula di Taylor al secondo ordine.


\begin{definition}
	$A \subseteq \R^n$ aperto, $f: A \rightarrow \R$ derivabile rispetto a $x_i$ in un intorno $U$ di $\overline{x_0}$. Se 
	\begin{align*} 
		\partial_{x_i}f: &U\rightarrow \R
		\\
		&\overline{x} \mapsto \partial_{x_i}f(\overline{x})
	\end{align*}
	
	è derivabile rispetto a $x_j$ in $\overline{x_0}$, allora si dice che $f$ ha derivata parziale seconda in $\overline{x_0}$
	
	$$\partial_{x_j}(\partial_{x_i}f)(\overline{x_0})$$
	
	ed è indicata con 
	
	$$\partial_{x_jx_i}^2f(\overline{x_0}), \qquad \frac{\partial^2 f}{\partial_{x_j}\partial_{x_i}}(\overline{x_0}), \qquad D_{x_jx_i}^2f(\overline{x_0})$$
	
	\begin{itemize}
		\item $f$ si dice derivabile due volte in $\overline{x_0} $ se in $\overline{x_0}$ ha tutte le $n^2$ derivate parziali seconde.
		
		\item $f$ si dice differenziabile due volte in $\overline{x_0}$ se è differenziabile in un intorno di $\overline{x_0}$ e le sue derivate parziali prime sono differenziabili in $\overline{x_0}$.
		
		\item $f$ si dice di classe $C^2$ in $A$ e si scrive $f \in C^2(A)$ se è derivabile due volte in $A$ e le sue derivate parziali seconde sono continue in $A$. In tal caso $f$ è differenziabile due volte in $A$ per il teorema differenziale totale. 
	\end{itemize}
\end{definition}


\begin{definition}
	In generale, se $A \subseteq \R^n$ è aperto e $f:A\rightarrow\R$, $f$ si dice di classe $C^k$ in $A$ e si scrive $f \in C^k(A)$, $k \geq 1$, se $f$ è derivabile $k$ volte in $A$ e le sue derivate parziali di ordine $k$ sono continue. $f$ si dice di classe $C^\infty$ in $A$ e si scrive $f \in C^\infty(A)$ se $f\in C^k(A) \,\, \forall \,\, k \geq 1$.
\end{definition}

\newpage %riq mal
\begin{exbar}
	$$f(x,y)=\sin(x^3y+y^3), \qquad f \in C^\infty(\R^2)$$
	
	$\partial_xf(x,y)=3x^2y \cos(x^3y+y^3)$
	
	$\partial_y f(x,y)=(x^3+3y^2)\cos(x^3y+y^3)$
	
	$\partial_y(\partial_xf)(x,y)=\partial_{yx}^2f(x,y)=3x^2 \cos(x^3y+y^3)-3x^2y(x^3+3y^2)\cos(x^3y+y^3)$
	
	$\partial_x(\partial_x f)(x,y)= \partial_{xx}^2f(x,y)=6xy\cos(x^3y+y^3)-3x^2y(3x^2y)\sin(x^3y+y^3)$
	
	$\partial_y(\partial_y f)(x,y)=\partial_{yy}^2f(x,y)=6y\cos(x^3y+y^3)-(x^3+3y)^2\sin(x^3y+y^3)$
	
	$\partial_x(\partial_y f)(x,y)=\partial_{xy}^2f(x,y)=3x^2\cos(x^3y+y^3)-3x^2y(x^3y+y^3)\sin(x^3y+y^3)$
	
	Noto che vale
	\begin{equation*}
		\partial_{yx}^2f(x,y)=\partial_{xy}^2f(x,y).
	\end{equation*}
	
	E' sempre vero? NO!
\end{exbar}


\subsubsection{Matrice Hessiana}
\begin{theorem} \textbf{(di Schwarz)}
	
	$A \subseteq\R^n$ aperto, $f:A\rightarrow\R$ tale che $\partial_{x_ix_j}^2f$ e $\partial_{x_jx_i}^2f$ ($i$ e $j$ fissati) esistono in un intorno $U$ di $\overline{x_0}\in A$ e sono continue in $\overline{x_0}$. Allora
	\begin{equation*}
		\partial_{x_jx_i}^2f(\overline{x_0})=\partial_{x_ix_j}^2f(\overline{x_0}).
	\end{equation*}
	
	In particolare, se $f \in C^2(A)$, allora 
	\begin{equation*}
		\partial_{x_jx_i}^2f(\overline{x})=\partial_{x_ix_j}^2f(\overline{x})\,\,\, \forall\,\, \overline{x} \in A\,\, \forall \,\,i,j=1,...,n 
	\end{equation*}  
\end{theorem}


\begin{exbar}
\begin{example}
	(funzione per cui le derivate seconde miste non sono uguali)
	
	$$f:\R^2\rightarrow\R$$
	
	\begin{equation*}
		f(x,y)=
		\begin{cases}
			xy\frac{x^2-y^2}{x^2+y^2}&\text{  se  }(x,y)\neq (0,0)
			\\
			0&\text{  se  }(x,y)=(0,0)
		\end{cases}
	\end{equation*}
	
	Calcoliamo $\partial_{yx}^2f(0,0)$ e $\partial_{xy}^2f(0,0)$
	
	$f(x,0)=f(0,y)\,\, \forall\,\, x,y \Rightarrow \nabla f(0,0)=(0,0)$
	
	$(x,y)\neq (0,0)$
	
	$\partial_x f(x,y)=y \frac{x^2-y^2}{x^2+y^2}+xy\frac{2x(x^2+y^2)-2x(x^2-y^2)}{(x^2+y^2)^2}= y \frac{x^2-y^2}{x^2+y^2}+4xy\frac{xy^2}{(x^2+y^2)^2}$
	
	$\partial_yf(x,y)=x\frac{x^2-y^2}{x^2+y^2}+xy\frac{-2y(x^2+y^2)-2y(x^2-y^2)}{(x^2+y^2)^2}=x\frac{x^2-y^2}{x^2+y^2}-4xy\frac{x^2y}{(x^2+y^2)^2}$
	
	$\partial_{yx}^2f(0,0)=\partial_y (\partial_x f)(0,0)=[\partial_y(\partial_xf(0,y))]|_{y=0}=[\partial_y(-y)]|_{y=0}=-1$
	
	$\partial_{xy}^2f(0,0)=\partial_x(\partial_yf)(0,0)= [\partial_x(\partial_yf(x,0))]|_{x=0}=[\partial_x(x)]|_{x=0}=1$
	
	Le due derivate parziali seconde miste in questo caso sono diverse.
\end{example}
\end{exbar}


\begin{definition}
	$A \subseteq \R^n$ aperto, $f: A \rightarrow \R$ derivabile due volte in $\overline{x}\in A$. Si dice \textbf{matrice hessiana} di $f$ in $\overline{x}$ la matrice
	\begin{equation*}
		D^2f(\overline{x})=
		\begin{pmatrix}
			\partial_{x_1x_1}^2f(\overline{x})& \partial_{x_1x_2}^2f(\overline{x})  & \cdots & \partial_{x_1x_n}^2f(\overline{x})
			\\
			\partial_{x_2x_1}^2f(\overline{x})& \partial_{x_2x_2}^2f(\overline{x}) & \cdots & \partial_{x_2x_n}^2f(\overline{x})
			\\
			\vdots&\vdots & \ddots &\vdots 
			\\
			\partial_{x_nx_1}^2f(\overline{x})& \partial_{x_nx_2}^2f(\overline{x})& \cdots & \partial_{x_nx_n}^2f(\overline{x}) 
		\end{pmatrix}
	\end{equation*}
\end{definition}


\begin{attbar}
	Se $f \in  C^2(A)$, $D^2f(\overline{x})$ è matrice simmetrica per il teorema di Schwarz.
\end{attbar}


\begin{exbar}
	$$f(x,y)=x^2y+y^3$$
	
	$$\partial_x f(x,y)=2xy, \qquad \partial_y f(x,y)=x^2+3y^2$$
	
	{\centering $\partial_{yx}^2f(x,y)=2x=\partial_{xy}^2f(x,y)$ per il teorema di Schwarz. \par}
	
	$$\partial_{xx}^2f(x,y)=2y, \qquad \partial_{yy}^2f(x,y)=6y$$
	
	$D^2f(1,1)=
	\begin{pmatrix}
		\partial_{xx}^2f(1,1) & \partial_{xy}^2f(1,1)
		\\
		\partial_{yx}^2f(1,1)& \partial_{yy}^2f(1,1)
	\end{pmatrix} =
	\begin{pmatrix}
		2 & 2 \\
		2 & 6
	\end{pmatrix}$
\end{exbar}


\subsubsection{Formula di Taylor}

$f: \R \rightarrow \R$ derivabile due volte in $x_0 \in \R$

$\Rightarrow f(x)=f(x_0)+f'(x_0)(x-x_0)+\frac{1}{2}f''(x_0)(x-x_0)^2+o(|x-x_0|^2)$

$f: \R^n \rightarrow \R$, differenziabile due volte in $\overline{x_0} \in \R^n $

\begin{align*} 
	\Rightarrow f(\overline{x})
	&=f(\overline{x_0})+\langle \nabla f(\overline{x_0}), \overline{x}-\overline{x_0} \rangle + \frac{1}{2} \langle D^2f(\overline{x_0})(\overline{x}-\overline{x_0}), \overline{x}-\overline{x_0} \rangle + o(\|\overline{x}-\overline{x_0}\|^2) =
	\\
	&=f(\overline{x_0})+\langle \nabla f(\overline{x_0}), \overline{x}-\overline{x_0} \rangle + \frac{1}{2} (\overline{x}-\overline{x_0})^TD^2 f(\overline{x_0})(\overline{x}-\overline{x_0})+o(\|\overline{x}-\overline{x_0}\|^2)
\end{align*}
\begin{attbar}
	\begin{equation*}
		T(\overline{x})=f(\overline{x_0}) +\langle\nabla 	f(\overline{x_0}),\overline{x}-\overline{x_0} \rangle +\frac{1}{2} \langle D^2f(\overline{x_0})(\overline{x} -\overline{x_0}),\overline{x}-\overline{x_0} \rangle
	\end{equation*}
\end{attbar}

polinomio di Taylor di ordine o grado $2$ di $f$ di centro $\overline{x_0}$.


\begin{theorem}
	
	\label{th: pag 377}
	$A \subseteq \R^n$ aperto, $f: A \rightarrow \R$
	\begin{enumerate}
		\item Se $f$ è differenziabile $2$ volte in $\overline{x_0} \in A$, allora
		
		\begin{equation}
			\label{eq: pag 377}
			f(\overline{x})=T(\overline{x})+o(\|\overline{x}-\overline{x_0}\|^2) \text{ per } \overline{x}\rightarrow \overline{x_0}
		\end{equation}
		
		e $T$ è l'unico polinomio di grado al più $2$ che soddisfa la formula di Taylor di ordine $2$ di $f$ di centro $\overline{x_0}$.
		
		\item Se $f$ è differenziabile due volte in $A$ e $[\overline{x_0},\overline{x}]\subseteq A$, $\overline{x_0},\overline{x}\in A$ allora $ \exists\,\, \overline{\xi}\in ]\overline{x_0},\overline{x}[$ tale che
		
		$$f(\overline{x})=f(\overline{x_0})+\langle \nabla f(\overline{x_0}),\overline{x} -\overline{x_0}\rangle +\frac{1}{2}\langle D^2 f(\overline{\xi})(\overline{x}-\overline{x_0}),\overline{x}-\overline{x_0} \rangle$$
	\end{enumerate} 
\end{theorem}


\begin{dembar}
	\textbf{Dimostrazione} dell'\textbf{Equazione \ref{eq: pag 377}}
	
	$\overline{x_0}, \overline{x}\in A$, $\overline{x}\neq \overline{x_0}$, $\phi:[0, \|\overline{x}-\overline{x_0}\|]\rightarrow \R$, $\overline{v}=\frac{\overline{x}-\overline{x_0}}{\|\overline{x}-\overline{x_0}\|}$
	
	$\phi(t)=f(\overline{x_0}+t\overline{v})$, restrizione di $f$ lungo il segmento $[\overline{x_0},\overline{x}]$.
	
	$\phi$ è derivabile due volte in $t=0$ perché $f$ lo è in $\overline{x_0}$.
	
	$$\phi(t)=\phi(0)+ {\color{teal} \phi'(0)} {\color{red}t} +\frac{1}{2} {\color{blue}\phi''(0)}t^2+o(t^2) \text{ per } t \rightarrow 0$$.
	
	{\color{red} $t=\| \overline{x}-\overline{x}_0 \|$}
	
	{ \color{teal}
		$\phi'(t)=\langle\nabla f(\overline{x_0}+t\overline{v}),\overline{v}\rangle$
		
		$\phi'(0)=\langle \nabla f(\overline{x_0}), \frac{\overline{x}-\overline{x_0}}{\|\overline{x}-\overline{x_0}\|}\rangle$
	}
	
	{\color{blue}
		$\phi''(t)=\langle D^2f(\overline{x_0}+t\overline{v})\overline{v},\overline{v} \rangle$
		
		$\phi''(0)=\langle D^2 f(\overline{x_0})\frac{\overline{x}-\overline{x_0}}{\|\overline{x}-\overline{x_0}\|}, \frac{\overline{x}-\overline{x_0}}{\|\overline{x}-\overline{x_0}\|}\rangle$
	}
	
	$$\lowercomment{\phi(\|\overline{x}-\overline{x_0}\|)}{f(\overline{x})}{}
	=\lowercomment{\phi(0)}{f(\overline{x}_0)}{} + \phi'(0) \|\overline{x}-\overline{x_0}\|+\frac{1}{2}\phi''(0)\| \overline{x}-\overline{x_0}\|^2+o(\|\overline{x}-\overline{x_0}\|^2)$$
	\begin{align*} 
		f(\overline{x})
		&=f(\overline{x_0})+
		\langle \nabla f(\overline{x_0}), \frac{\overline{x}-\overline{x_0}}{\cancel{\|\overline{x}-\overline{x_0}\|}} \rangle \cancel{\|\overline{x} -\overline{x_0}\|} +
		\\
		&\qquad + \frac{1}{2}\langle D^2 f(\overline{x_0})\frac{\overline{x}-\overline{x_0}}{\cancel{\|\overline{x}-\overline{x_0}\|}},\frac{\overline{x}-\overline{x_0}}{\cancel{\|\overline{x}-\overline{x_0}\|}}  \rangle \cancel{\|\overline{x}-\overline{x_0}\|^2}
		+o(\|\overline{x}-\overline{x_0}\|)
	\end{align*}
\end{dembar}


\begin{exbar}
\begin{example}
	$$f(x,y)=e^y\ln(1+x)$$
	
	Calcoliamo il polinomio di Taylor di $f$ di ordine $2$ e centro $(0,0)$.
	
	$$\partial_xf(x,y)=\frac{e^y}{1+x} \qquad \partial_y f(x,y) =e^y \ln(1+x)$$
	
	$$\partial_xf(0,0)=1 \qquad \partial_y f(0,0)=0$$
	
	$$\partial_{xx}^2f(x,y)=\frac{-e^y}{(1+x)^2} \qquad \partial_{yy}^2f(x,y)=e^y\ln(1+x)$$
	
	{\centering $\partial_{yx}^2 f(x,y)=\frac{e^y}{1+x}=\partial_{xy}^2f(x,y)$ per il teorema di Schwarz \par}
	
	$$\partial_{xx}^2f(0,0)=-1 \qquad \partial_{yy}^2f(0,0)=0 \qquad \partial_{yx}^2f(0,0)=1=\partial_{xy}^2f(0,0)$$
	
	$$\nabla f(0,0)=(1,0)$$
	
	$$D^2f(0,0)=
	\begin{pmatrix}
		-1&1
		\\
		1&0
	\end{pmatrix}$$
	\begin{align*} 
		T(\overline{x})&=f(0,0)+\langle\nabla f(0,0),(x,y) \rangle + \frac{1}{2}\langle D^2f(0,0)(x,y),(x,y) \rangle=
		\\
		&=0+\langle (1,0),(x,y) \rangle + \frac{1}{2}
		\begin{pmatrix}
			x&y
		\end{pmatrix}\begin{pmatrix}
			-1& 1\\
			1 &0
		\end{pmatrix} \begin{pmatrix}
			x\\
			y
		\end{pmatrix}=
		\\
		&=x+\frac{1}{2}
		\begin{pmatrix}
			-x+y & x
		\end{pmatrix}\begin{pmatrix}
			x\\
			y
		\end{pmatrix}=
		\\
		&=x-\frac{1}{2}x^2+\frac{1}{2}xy+\frac{1}{2}xy=
		\\
		&=x-\frac{1}{2}x^2+xy
	\end{align*}
	
	$$f(x,y)=e^y\ln(1+x)$$
	
	$e^y=1+y+\frac{1}{2}y^2+o(y^2)$ per $y \rightarrow0$
	
	$\ln(1+x)=x-\frac{x^2}{2}+o(x^2)$ per $x \rightarrow 0$
	
	
	\begin{align*} 
		f(x,y)
		&=(1+y+\frac{1}{2}y^2+o(y^2))(x-\frac{x^2}{2}+o(x^2))=
		\\
		&=x-\frac{x^2}{2}+o(x^2)+xy-\cancel{\frac{x^2}{2}y}+o(x^2y)+\cancel{\frac{1}{2}x^2y}-\frac{1}{4}x^4y^4+o(x^2y^2)+o(y^2)=
		\\
		&=x-\frac{x^2}{2}+xy+\bigg[ \lowercomment{-\frac{1}{4}x^4y^4}{o(x^2)}{}+o(x^2)+o(y^2) \bigg]
	\end{align*}
	
		$$o(x^2)=o(x^2+y^2) \qquad o(y^2)=o(x^2+y^2)$$
		
		{\color{teal}
			$\lim_{(x,y)\rightarrow(0,0)}\frac{o(x^2)}{x^2+y^2}=0$, è vero? 
			
			Sì: $o(x^2)=x^2\cdot \uppercomment{o(1)}{\to 0}{\text{per }x\to0}$
			
			$\lim_{(x,y)\rightarrow(0,0)} \frac{o(x^2)}{x^2+y^2}=\lim_{(x,y)\rightarrow(0,0)} \uppercomment{\frac{x^2}{x^2+y^2}}{}{\leq1}\cdot o(1)=0$
		}
			
		\begin{equation} 
			\label{eq: pag 381}
			f(x,y)=x-\frac{x^2}{2}+xy+o(x^2+y^2)=x-\frac{x^2}{2}+xy+o(\|x,y\|^2)
		\end{equation}
		
		Il polinomio di Taylor di $f$ di ordine $2$ e centro $(0,0)$ è l'unico polinomio di grado al più $2$ che soddisfa l'equazione \ref{eq: pag 381} 
		
		$$\Rightarrow T(\overline{x})=x-\frac{x^2}{2}+xy$$
\end{example}
\end{exbar}


\subsubsection{Studio della natura dei punti critici}

Sia $Q: \R^n$ forma quadratica. $Q(\overline{x})=\overline{x}^T A \overline{x}$, con $A$ matrice simmetrica $n\times n$.

\begin{definition}
	\begin{enumerate}
		\item $Q$ si dice \textbf{definita positiva} se $Q(\overline{x}) >0\,\, \forall\,\, \overline{x} \neq \overline{0}$.
		
		\item $Q$ si dice \textbf{definita negativa} se $Q(\overline{x})<0 \,\, \forall \,\, \overline{x} \neq \overline{0}$.
		
		\item $Q$ si dice \textbf{semidefinita positiva} se $Q(\overline{x})\geq 0 \,\, \forall \,\, \overline{x} \in \R^n$.
		
		\item $Q$ si dice \textbf{semidefinita negativa} se $Q(\overline{x})\leq 0 \,\, \forall \,\, \overline{x} \in \R^n$.
		
		\item $Q$ si dice \textbf{indefinita} se $\exists \,\,\overline{x_1},\overline{x_2}\in \R^{n}$ tali che $Q(\overline{x_1})>0$ e $Q(\overline{x_2})<0$.
	\end{enumerate}
\end{definition}


\begin{proposition}
	$Q$ forma quadratica su $\R^n$. Sia $A$ la matrice simmetrica $n \times n$ associata a $Q$ e siano $\lambda_1,...,\lambda_n$ gli autovalori di $A$ contati con la loro molteplicità.
	\begin{enumerate}
		\item $Q$ è definita positiva $\Leftrightarrow \lambda_i >0 \,\, \forall i =1,...,n$ e in tal caso $\exists \,\, c >0 \mid Q(\overline{x})\geq c\|\overline{x}\|^2\,\, \forall \overline{x} \in \R^n$.
		\item $Q$ è semidefinita positiva $\Leftrightarrow \lambda_i \geq 0 \,\, \forall i=1,...,n$.
		\item $Q$ è definita negativa $\Leftrightarrow \lambda_i <0 \,\, \forall i =1,...,n$ e in tal caso $\exists \,\, c >0 \mid Q(\overline{x})\leq -c\|\overline{x}\|^2\,\, \forall \overline{x} \in \R^n$.
		\item $Q$ è semidefinita negativa $\Leftrightarrow \lambda_i \leq 0 \,\, \forall i=1,...,n$.
		\item $Q$ è indefinita $\Leftrightarrow \exists\,\, i,j \in \{1,...,n\} \mid \lambda_i <0$ e $\lambda_j>0$.
	\end{enumerate}
\end{proposition}


\begin{attbar}
	Se $Q$ è la forma quadratica nulla, è contemporaneamente semidefinita positiva e semidefinita negativa.
\end{attbar}

Sia $A$ matrice $n \times n$ nella forma
\begin{equation*}
	A= 
	\begin{pmatrix}
		a_{11}&a_{12} &\cdots &a_{1n} \\
		a_{21}&a_{22} &\cdots &a_{2n} \\
		\vdots&\vdots &\ddots &\vdots \\
		a_{n1}&a_{n2} &\cdots &a_{nn} \\
	\end{pmatrix}
\end{equation*}

I minori principali di $A$ sono i determinanti delle sottomatrici lungo la diagonale principale cioè $\alpha_1=a_{11}$, $\alpha_2=\det\begin{pmatrix}
	a_{11}&a_{12} \\
	a_{21}&a_{22}
\end{pmatrix}$, $\alpha_3=\det\begin{pmatrix}
	a_{11}&a_{21}&a_{31} \\
	a_{21}&a_{22}&a_{32} \\
	a_{31}&a_{23}&a_{33} 
\end{pmatrix}$
etc.

\image{calcolo_differenziale/pag384.png}

\begin{proposition}
	Sia $Q$ forma quadratica, sia $A$ la matrice simmetrica ad essa associata e sia $\alpha_1,...,\alpha_n$ i suoi minori principali.
	\begin{enumerate}
		\item $Q$ è definita positiva $\Leftrightarrow \alpha_i >0\,\, \forall \,\, i =1,...,n$
		\item $Q$ è definita negativa $\Leftrightarrow \alpha_i <0 $ per i dispari e $\alpha_1>0$ per i pari. 
	\end{enumerate}
	
	In particolare, se $A$ è matrice $2 \times 2$, allora
	\begin{enumerate}
		\item se $\det A <0 \Rightarrow Q$ è indefinita,
		\item se $\det A=0 \Rightarrow 0$ è semidefinita.
	\end{enumerate}
\end{proposition}


\begin{attbar}
	\textbf{Notazione.} Con abuso di notazione si usa dire:
	\begin{enumerate}
		\item $Q$ definita positiva: $A >0$; 
		\item $Q$ definita negativa: $A<0$;
		\item $Q$ semidefinita positiva: $A \geq 0$;
		\item $Q$ semidefinita negativa: $A \leq 0$.
	\end{enumerate}
\end{attbar}

%======================================================================

\begin{theorem}
	
	\label{th: pag 385}
	$A \subseteq \R^n$ aperto, $f:A \rightarrow  \R$ differenziabile due volte in $\overline{x_o}\in A$, punto stazionario per $f$.
	\begin{enumerate}
		\item Se $D^2f(\overline{x_0})>0 ${\color{blue} (definita positiva)}, allora $\overline{x_0}$ è punto di minimo relativo per $f$.
		\item Se $D^2f(\overline{x_0})<0${\color{blue} (definita negativa)}, allora $\overline{x_0}$ è punto di massimo relativo per $f$. 
		\item Se $D^2f(\overline{x_0})$ è indefinita, allora $\overline{x_0}$ è punto di sella per $f$.
		\item Se $\overline{x_0}$ è punto di minimo relativo, allora $D^2f(\overline{x_0})\geq0$ {\color{blue}(semidefinita positiva)}.
		\item Se $\overline{x_0}$ è punto di massimo relativo per $f$, allora $D^2f(\overline{x_0})\leq0$ {\color{blue}(semidefinita negativa)}.
	\end{enumerate}
\end{theorem}


\begin{attbar}
	I punti 4. e 5. sono conseguenza di 1., 2. e 3.
\end{attbar}


\begin{exbar}
	$$f(x,y)=x^4+y^4$$
	
	$f(x,y)\geq 0 = f(0,0)$ $\forall (x,y)\in \R^2 \Rightarrow (0,0)$ è punto di minimo assoluto.
	
	$$\partial_xf(x,y)=4x^3 \qquad \partial_yf(x,y)=4y^3$$
	
	$$\partial_{xx}^2f(x,y)=12x^2 \quad \partial_{yy}^2f(x,y)=12y^2 \quad \partial_{xy}^2f(x,y)=\partial_{yx}^2f(x,y)=0$$
	
	{\centering $D^2f(0,0)=\begin{pmatrix}
		0&0\\
		0&0
	\end{pmatrix}$
	che è semidefinita. \par}
\end{exbar}


\begin{dembar}
	\textbf{Dimostrazione} del \textbf{Teorema \ref{th: pag 385}}
	
	\begin{enumerate}
		\item 
		\begin{align*} 
			f(\overline{x})&=f(\overline{x_0})+ \overbrace{\langle \nabla f(\overline{x_0}), \overline{x}-\overline{x_0} \rangle}^{{\color{blue} =0 \text{ perché } \nabla f(\overline{x}_0=\overline{0})}} +\frac{1}{2}\langle D^2f(\overline{x_0})(\overline{x}-\overline{x_0}),(\overline{x}-\overline{x_0}) \rangle+o(\| \overline{x}-\overline{x_0} \|^2)=
			\\
			&=f(\overline{x_0})+\frac{1}{2}\langle D^2f(\overline{x_0})(\overline{x}-\overline{x_0}),(\overline{x}-\overline{x_0})\rangle+o(\|\overline{x}-\overline{x_0}\|^2)
		\end{align*}
		
		$D^2f(\overline{x_0})>0 \Rightarrow \exists \,\, c >0 \mid$
		
		$$\langle D^2f(\overline{x_0})(\overline{x}-\overline{x_0}),\overline{x}-\overline{x_0} \rangle \geq \|\overline{x}-\overline{x_0} \|^2$$
		
		$$\Rightarrow f(\overline{x})\geq f(\overline{x_0})+\frac{c}{2}\|\overline{x}-\overline{x_0}\|^2+o(\|\overline{x}-\overline{x_0}\|^2)=-f(\overline{x_0})+\|\overline{x}-\overline{x_0} \|^2(\frac{c}{2}+\uppercomment{o(1)}{=0}{\text{per } \overline{x} - \overline{x}_0})$$
		
		$\Rightarrow \exists\,\, r>0 \mid \frac{c}{2}+o(1)\geq \frac{c}{4}$in $B_r(\overline{x_0})$ per il teorema della permanenza del segno
		
		$\Rightarrow \forall\,\, \overline{x}\in B_r(\overline{x_0})$ si ha
		
		$$f(\overline{x})\geq f(\overline{x_0})+\frac{c}{4}\|\overline{x}-\overline{x_0}\|^2 \geq f(\overline{x_0}) \Rightarrow \overline{x_0}$$
		
		è punto di minimo relativo.
		
		\item Analogo al punto 1. Si sfrutta il fatto che $\exists \,\, c>0 \mid$
		
		$$ \langle D^2f(\overline{x_0})(\overline{x}-\overline{x_0}),\overline{x}-\overline{x_0} \rangle \leq -c\|\overline{x}-\overline{x_0}\|^2$$
		
		\item $D^2f(\overline{x_0})$ è indefinita, quindi ha due autovalori $\lambda_1$ e $\lambda_2$ tali che $\lambda_1<0$ e $\lambda_2>0$.
		
		Siano $\overline{r_1}$ e $\overline{r_2}$ i rispettivi autovettori tali che 
		
		$$D^2f(\overline{x_0})\overline{r_1}=\lambda_1\overline{r_1}, \qquad D^2f(\overline{x_0})=\lambda_2\overline{r_2}$$
		
		 che supponiamo avere norma uno, $\|\overline{r_1} \|=\|\overline{r_2}\|=1$.
		
		$$f(\overline{x})=f(\overline{x_0})+\frac{1}{2}\langle D^2f(\overline{x_0})(\overline{x}-\overline{x_0}),\overline{x}-\overline{x_0} \rangle + o(\|\overline{x}-\overline{x_0}\|^2)$$
		
		$\overline{x}=\overline{x_0}+t\overline{r_1}$, $t \in \R$, sufficientemente piccolo in modo che $\overline{x_0}+t\overline{r_1}\in A$.
		\begin{align*} 
			f(\overline{x_0}+t\overline{r_1})
			&=f(\overline{x_0})+\frac{1}{2}\langle D^2f(\overline{x_0})t\overline{r_1},t\overline{r_1} \rangle + o(t^2)=
			\\
			&=f(\overline{x_0})+\frac{1}{2}\langle \lambda_1 +\overline{r_1},t\overline{r_1}\rangle + o(t^2)=
			\\
			&=f(\overline{x_0})+\frac{1}{2}\lambda_1 t^2 + o(t^2)=f(\overline{x_0})+t^2(\frac{1}{2} \lowercomment{\lambda_1}{<0}{}+\uppercomment{o(1)}{0 \text{ per } t\to0 \text{ cioè}}{\text{per } \overline{x}_0 + t\overline{r}_1\to\overline{x}_0}) 
		\end{align*}
		
		{\centering $\Rightarrow \frac{1}{2}\lambda_1+o(1) < \frac{1}{4}\lambda_1$ in un intorno di $t=0$ \par}
			
		$f(\overline{x_0}+t\overline{r_1})<f(\overline{x_0})+\frac{1}{4}\lambda_1 t^2< f(\overline{x_0})$ in un intorno di $t=0$, $t\neq 0$.
		
		Analogamente
		\begin{align*} 
			f(\overline{x_0}+t\overline{r_2})
			&=f(\overline{x_0})+\frac{1}{2}\langle D^2f(\overline{x_0})t\overline{r_2},t\overline{r_1} \rangle +o(t^2)=
			\\
			=f(\overline{x_0})+\frac{1}{2}\lambda_2t^2+o(t^2)>f(\overline{x_0})
		\end{align*}
		
		in un intorno di $t=0$, $t \neq 0$
		
		$\Rightarrow \overline{x_0}$ non è punto di estremo per $f \Rightarrow \overline{x_0}$ è punto di sella per $f. \qquad \square$
	\end{enumerate}
\end{dembar}


\begin{exbar}
\begin{example}
	Cercare i punti di estremo relativo della funzione 
	$$f(x,y)=4xe^x\cos y, \quad (x,y)\in\R^2, \quad f \in C^\infty(\R^2)$$
	
	Cerchiamo i punti critici di $f$ 
	
	$$\partial_xf(x,y)=4e^x(x+1)\cos y \qquad \partial_yf(x,y)=-4xe^x\sin y$$
	
	Deve essere 
	\begin{gather*} 
		\nabla f(x,y)=(0,0) \Leftrightarrow 
		\begin{cases}
			\partial_x f(x,y)=0\\
			\partial_y f(x,y)=0
		\end{cases} = 
		\begin{cases}
			e^x(x+1)\cos y=0\\
			xe^x\sin y=0
		\end{cases}
		\\
		\Leftrightarrow 
		\begin{cases}
			x=-1 &\text{  o  }\cos y =0\\
			x=0 &\text{  o  }\sin y=0
		\end{cases} \Leftrightarrow 
		\begin{cases}
			x=-1\\
			\sin y =0
		\end{cases} \text{ oppure } 
		\begin{cases}
			\cos y=0\\
			x=0
		\end{cases}
		\\
		(-1,k\pi), \quad (0,\frac{\pi}{2}+k\pi), \quad k \in \mathbb{Z}
	\end{gather*}
	
	Studiamo la natura dei punti critici.
	
	$$\partial_xf(x,y)=4e^x(x+1)\cos y \qquad \partial_yf(x,y)=-4xe^x\sin y$$
	
	$$\quad \partial_{xx}^2f(x,y)=4e^x(x+2)\cos y \quad \partial_{yy}^2f(x,y)=-4xe^x \cos y$$
	
	$$ \partial_{yx}f(x,y)=-4e^x(x+1)\sin y= \partial_{xy}^2f(x,y)$$
	
	$$D^2f(x,y)=
	4 \begin{pmatrix}
		e^x(x+2)\cos y & -e^x(x+1)\sin y\\
		-e^x(x+1)\sin y & -x e^x \cos y
	\end{pmatrix}$$
	
	$$D^2f(-1, k\pi)=
	4 \begin{pmatrix}
		e^{-1}\cos(k \pi)&0\\
		0&e^{-1}\cos(k\pi)
	\end{pmatrix}=\begin{pmatrix}
		e^{-1}(-1)^k&0\\
		0&e^{-1}(-1)^k
	\end{pmatrix}$$
	
	$k$ pari $D^2f(-1,k\pi) >0 \Rightarrow (-1,k\pi)$ sono punti di minimo relativo.
	
	$k$ dispari $D^2f(-1,k\pi)< 0 \Rightarrow (-1,k\pi)$ sono punti di massimo relativo.
	
	$$D^2f(0,\frac{\pi}{2}+k \pi)=4 
	\begin{pmatrix}
		0 & -\sin(\frac{\pi}{2}+k\pi)\\
		-\sin(\frac{\pi}{2}+k\pi)& 0
	\end{pmatrix}=
	4 \begin{pmatrix}
		0&(-1)^{k+1}\\
		(-1)^{k+1}&0
	\end{pmatrix}$$
	
	che è matrice indefinita $\Rightarrow$ i punti $(0,\frac{\pi}{2}+k \pi)$ sono punti di sella.
\end{example}
\end{exbar}


\begin{exbar}
\begin{example}
	Calcolare i punti critici della funzione 
	
	$$f(x,y)=(y-x^2)(y-2x^2), \qquad (x,y) \in \R^2$$
	
	e studiare la loro natura. $f\in C^\infty (\R^2)$.
	
	$\partial_xf(x,y)=-2x(y-2x^2)-4x(y-x^2)=8x^3-6xy$
	
	$\partial_yf(x,y)=y-2x^2+y-x^2=2y-3x^2$
	
	$$\begin{cases}
		\partial_xf(x,y)=0\\
		\partial_yf(x,y)=0
	\end{cases} \Leftrightarrow 
	\begin{cases}
		8x^3-6xy=0\\
		2y-3x^2=0
	\end{cases} \Leftrightarrow 
	\begin{cases}
		x=0 \text{   o   } 4x^2-3y=0\\
		y=\frac{3}{2}x^2
	\end{cases}\Leftrightarrow 
	\begin{cases}
		x=0\\
		y=0
	\end{cases}$$
	
	$\Rightarrow(0,0)$ è l'unico punto critico per $t$
	
	Calcoliamo $D^2f(0,0)$.
	
	$$\partial_x f(x,y) = 8x^3-6xy \qquad \partial_y f(x,y) = 2y-3x^2$$
	
	$$\partial_{xx}^2f(x,y)=24x^2-6y \quad \partial_{yy}^2f(x,y)=2 \quad \partial_{yx}^2f(x,y)=\partial_{xy}^2f(x,y)-6x$$
	
	$$D^2f(0,0)=\begin{pmatrix}
		0&0\\
		0&2
	\end{pmatrix} \Rightarrow$$
	
	è semidefinita positiva e non è la matrice nulla $\Rightarrow (0,0)$ è punto di minimo o è punto di sella.
	
	$$f(x,y)=(y-x^2)(y-2x^2)$$
	
	$$f(0,0)=0, \quad f(x,y) \big|_{y=x^2}=0 \quad f(x,y)\big|_{y=2x^2}=0$$
	
	\image{calcolo_differenziale/pag394.png} % pag 394

	Se $y < x^2 \Rightarrow y-x^2 < 0$ e $y-2x^2< 0 \Rightarrow f(x,y)>0$. 
	
	Se $y > 2x^2 \Rightarrow y-x^2 >0$ e $y-2x^2 >0 \Rightarrow f(x,y)> 0$.
	
	$$x^2< y< 2x^2$$
	
	$y- x^2>0$, $y-2x^2< 0 \Rightarrow f(x,y)<0 \Rightarrow (0,0)$ è punto di sella.
\end{example}
\end{exbar}


\begin{exbar}
\begin{example}
	Studiare la natura dei punti critici della funzione $f: \R^3 \rightarrow \R$ definita da 
	
	$$f(x,y,z)=x\sin z + y^2$$
	
	$$\partial_x f(x,y,z)=\sin z \quad \partial_y f(x,y,z)=2y \quad \partial_z f(x,y,z)=x \cos z$$
	
	$$\begin{cases}
		\partial_x f(x,y,z)=0\\
		\partial_y f(x,y,z)=0\\
		\partial_z f(x,y,z)=0
	\end{cases}\Leftrightarrow
	\begin{cases}
		\sin z=0\\
		2y=0\\
		x\cos z=0
	\end{cases}\Leftrightarrow
	\begin{cases}
		z=k\pi\\
		y=0\\
		x=0 \text{  o  } z=\frac{\pi}{2}+k\pi
	\end{cases}$$
	
	$$(0,0,k\pi), \qquad k \in \mathbb{Z}$$
	
	Per studiarne la natura calcoliamo le derivate seconde.
	
	$$\partial_x f(x,y,z)=\sin z \quad \partial_y f(x,y,z)=2y \quad \partial_z f(x,y,z)=x \cos z$$
	
	$$\partial_{xx}^2f(x,y,z)=0 \quad \partial_{yx}^2f(x,y,z)=0=\partial_{xy}^2f(x,y,z) \quad \partial_{zx}^2f(x,y,z)=\cos z$$
	
	$$\partial_{yy}^2f(x,y,z)=2 \quad \partial_{zy}^2f(x,y,z)=0=\partial_{yz}^2f(x,y,z) \quad \partial_{zz}^2f(x,y,z)=-x \sin z$$
	
	$$D^2f(x,y,z)=\begin{pmatrix}
		0&0&\cos z\\
		0&2&0\\
		\cos z&0&-x \sin z
	\end{pmatrix}$$
	
	$$D^2f(0,0,k\pi)=\begin{pmatrix}
		0&0&(-1)^k\\
		0&2&0\\
		(-1)^k&0&0
	\end{pmatrix}$$
	
	Calcoliamo gli autovalori della matrice hessiana.
	
	$0=\det(D^2 f(0,0,k\pi)-\lambda I)= 
	\det\begin{pmatrix}
		-\lambda&0&(-1)^k\\
		0&-2\lambda&0\\
		(-1)^k&0&-\lambda
	\end{pmatrix}=(2-\lambda)(\lambda^2-1)$
	
	Autovalori: $\lambda=2$ e $\lambda= \pm 1$. Avendo due autovalori di segno opposto, la matrice hessisiana è indefinita e quindi tutti i punti critici sono punti di sella.
\end{example}
\end{exbar}


\begin{exbar}
\begin{example}
	Studiare i punti critici della funzione definita da 
	
	$$f(x,y,z)=2x^3-2y^2+z^2+3(x^2+y)z, \quad (x,y,z)\in \R^3 \quad f\in C^\infty(\R^3)$$.
	
	$$\partial_xf(x,y,z)=6x^2+6xz \quad \partial_yf(x,y,z)=-4y+3z \quad \partial_zf(x,y,z)=2z+3(x^2+y)$$
	
	$\nabla f(x,y,z)=(0,0,0)$
	
	$$\begin{cases}
		6x(x+z)=0\\
		-4y+3z=0\\
		2z+3(x^2+y)=0
	\end{cases}\Leftrightarrow
	\begin{cases}
		x=0\text{  o  }x=-z\\
		-4y+3z=0\\
		2z+3(x^2+y)=0
	\end{cases}\Leftrightarrow
	\begin{cases}
		x=0\\
		-4y+3z=0\\
		3y+2z=0
	\end{cases}$$
	
	$$\det\begin{pmatrix}
		-4&3\\
		3&2
	\end{pmatrix}=-17\neq 0 \Leftrightarrow (x,y,z)=(0,0,0)$$
	
	$$\begin{cases}
		x=-z\\
		-4y+3z=0\\
		2z+3z^2+3y=0
	\end{cases}\Leftrightarrow
	\begin{cases}
		x=-z\\
		y=\frac{3}{4}z\\
		2z+3z^2+\frac{9}{4}z=0
	\end{cases}$$
	$\Leftrightarrow (x,y,z)=(0,0,0)$ oppure $(x,y,z)=(\frac{17}{12},-\frac{17}{16},-\frac{17}{12})$.
	
	I punti critici sono $(0,0,0)$ e $(\frac{17}{12},-\frac{17}{16},-\frac{17}{12})$. Studiamone la natura.
	
	$$\partial_xf(x,y,z)=6x^2+6xz \quad \partial_yf(x,y,z)=-4y+3z \quad \partial_zf(x,y,z)=2z+3(x^2+y)$$
	
	$$\partial_{xx}^2f(x,y,z)=12x+6z \quad \partial_{yx}^2f(x,y,z)=0=\partial_{xy}^2f(x,y,z) \quad \partial_{zx}^2f(x,y,z)=6x=\partial_{xz}^2f(x,y,z)$$
	
	$$\partial_{yy}^2f(x,y,z)=-4 \quad \partial_{zy}^2f(x,y,z)=3=\partial_{yz}^2f(x,y,z) \quad \partial_{zz}^2f(x,y,z)=2$$
	
	$$D^2f(x,y,z)=\begin{pmatrix}
		12x+6z&0&6x\\
		0&-4&3\\
		6x&3&2
	\end{pmatrix}$$
	
	$$D^2f(0,0,0)=\begin{pmatrix}
		0&0&0\\
		0&-4&3\\
		0&3&2
	\end{pmatrix}$$
	
	\begin{align*} 
		0
		&=\det(D^2f(0,0,0)-\lambda I)
		= \det \begin{pmatrix}
			-\lambda&0&0\\
			0&-4-\lambda&3\\
			0&3&2-\lambda
		\end{pmatrix}=
		\\
		&=-\lambda[(\lambda-2)(4+\lambda)-9]=-\lambda[\lambda^2+2\lambda-17]
	\end{align*}
	
	$ \Rightarrow$ ha autovalore positivo ad uno negativo $\Rightarrow$ la metrica è indefinita $\Rightarrow (0,0,0)$ è punto di sella.
	
	$$D^2f(\frac{17}{12},-\frac{17}{16},-\frac{17}{12})=\begin{pmatrix}
		\frac{17}{2}&0&\frac{17}{2}\\
		0&-4&3\\
		\frac{17}{2}&3&2
	\end{pmatrix}$$
	
	$\det D^2f(\frac{17}{12},-\frac{17}{16},-\frac{17}{12})\neq 0$ e i primi due minori principali $\alpha_1$ e $\alpha_2$ soddisfano $\alpha_1= \frac{17}{2} >0$ e $\alpha_2 =-34 <0$
	
	$\Rightarrow$ la matrice è indefinita $\Rightarrow (\frac{17}{12},-\frac{17}{16},-\frac{17}{12})$ è punto di sella.
\end{example}
\end{exbar}


\subsection{Matrice Jacobiana}

$A \subseteq \R^n$ aperto, $\overline{f}: A \rightarrow \R^p$, $p \geq 1$, 

$$\overline{f}(\overline{x})=(f_1(\overline{x}),f_2(\overline{x}),...,f_p(\overline{x}))$$

$f_i:A \rightarrow \R$, $ i =1,...,p$.


\begin{definition}
	$\overline{f}$ si dice derivabile in $\overline{x_0}$ nella direzione $\overline{v}\in \R^n$, $\overline{v}$ versore, se tali sono $f_1,...,f_p$ e in tal caso si pone
	\begin{equation*}
		D_{\overline{v}}f(\overline{x_0})=
		\begin{pmatrix}
			D_{\overline{v}}f_1(\overline{x_0})\\
			D_{\overline{v}}f_2(\overline{x_0})\\
			\vdots\\
			D_{\overline{v}}f_p(\overline{x_0})
		\end{pmatrix}
	\end{equation*}
	
	In particolare, $\overline{f}$ è derivabile in $\overline{x_0}$ se ciascuna componente $f_1,...,f_p$ lo è e in tal caso si pone 
	\begin{equation*}
		\partial_{x_k}\overline{f}(\overline{x_0})=
		\begin{pmatrix}
			\partial_{x_k}f_1(\overline{x_0})\\
			\partial_{x_k}f_2(\overline{x_0})\\
			\vdots\\
			\partial_{x_k}f_p(\overline{x_0})
		\end{pmatrix}
	\end{equation*}
\end{definition}


\begin{definition}
	$\overline{f}$ è differenziabile in $\overline{x_0}\in A  \Leftrightarrow$ lo sono tutte le sue componenti. Se $\overline{f}$ è derivabile in $\overline{x_0}$, poniamo 
	\begin{equation*}
		D\overline{f}(\overline{x_0})=\overline{d_f}(\overline{x_0})=
		\begin{pmatrix}
			\nabla f_1(\overline{x_0})\\
			\nabla f_2(\overline{x_0})\\
			\vdots\\
			\nabla f_p(\overline{x_0})
		\end{pmatrix}=
		\begin{pmatrix}
			\partial_{x_1}f_1(\overline{x_0})&\partial_{x_2}f_1(\overline{x_0})&\cdots&\partial_{x_n}f_1(\overline{x_0})\\
			\partial_{x_1}f_2(\overline{x_0})&\partial_{x_2}f_2(\overline{x_0})&\cdots&\partial_{x_n}f_2(\overline{x_0})\\
			\vdots&\vdots&\ddots&\vdots \\
			\partial_{x_1}f_p(\overline{x_0})&\partial_{x_2}f_p(\overline{x_0})&\cdots&\partial_{x_n}f_p(\overline{x_0})
		\end{pmatrix}
	\end{equation*}
	
	che viene detta \textbf{matrice jacobiana} di $\overline{f}$ in $\overline{x_0}$.
\end{definition}


\begin{theorem}
	Se $\overline{f}$ è differenziabile in $\overline{x_0}$, allora $\overline{f}(\overline{x})=\overline{f}(\overline{x_0})+Df(\overline{x_0})(\overline{x}-\overline{x_0})+o(\|\overline{x}-\overline{x_0}\|)$.
	
	{\color{teal}
		$f: \R \rightarrow \R$, differenziabile in $x_0$, 
		
		$$f(x)=f(x_0)+\lowercomment{f'(x_0)}{\in\R}{}(x-x_0)+o(x-x_0)$$
		
		$f:\R^n \rightarrow \R$ differenziabile in $\overline{x_0}$,
		
		$$f(\overline{x})=f(\overline{x_0})+\langle \lowercomment{\nabla f}{\in\R^n}{}(\overline{x_0}), \overline{x}-\overline{x_0} \rangle + o(\|\overline{x}-\overline{x_0} \|)$$
		
		$\overline{f}: \R^n \rightarrow \R^p$ differenziabile in $\overline{x_0}$
		
		$$\overline{f}(\overline{x})=\overline{f}(\overline{x_0})+ \lowercomment{Df(\overline{x_0})}{\text{matrice } p\times n}{}(\overline{x}-\overline{x_0})+o(\|\overline{x}-\overline{x_0}\|)$$
		}
\end{theorem}


\begin{exbar}
	$\overline{f}:\R^n \rightarrow \R^p$ lineare, $\overline{f}(\overline{x})=M \overline{x}$, $M$ matrice $p \times n$, allora $D \overline{f}(\overline{x})=M$.
\end{exbar}


\begin{exbar}
	$\overline{f}:\R^2 \rightarrow \R^3$
	
	$$\overline{f}(x,y)=
	\begin{pmatrix}
		\sin(xy)\\
		x^2+y^2\\
		e^{x^2y^3}
	\end{pmatrix}$$
	
	$$Df(x,y)=
	{\color{blue} 
		\begin{pmatrix}
			\nabla \sin(xy)\\
			\nabla (x^2+y^2)\\
			\nabla (e^{x^2y^3})
		\end{pmatrix}=
	}
	\begin{pmatrix}
		y \cos(xy) & x \cos (xy)\\
		2x & 2y\\
		2xy^3e^{x^2y^3}&3x^2y^2e^{x^2y^3}
	\end{pmatrix}$$
\end{exbar}


\begin{theorem} \textbf{Regola della catena}
	
	$A \subseteq \R^n$ e $B \subseteq \R^p$ aperti, $\overline{f}: B \rightarrow \R^k$ e $\overline{g}:A \rightarrow B$
	{\color{blue}($\overline{f}\circ \overline{g}: A \rightarrow \R^k$)}.
	
	Se $\overline{g}$ è differenziabile in $\overline{x_0}\in A$ e $\overline{f}$ è differenziabile in $\overline{g}(\overline{x_0})\in B$, allora $\overline{f}\circ \overline{g}$ è differenziabile in $\overline{x_0}$ e in tal caso
	\begin{equation*}
		D(\overline{f}\circ \overline{g})(\overline{x_0})=D\overline{f}(\overline{g}(\overline{x_0}))D\overline{g}(\overline{x_0}).
	\end{equation*}	
\end{theorem}

% PAG 405